\noindent Definir o mundo do jogo não é uma tarefa trivial, uma vez que os jogos são projetos multidisciplinares que envolvem narrativa, mecânicas, arte, programação e design de experiência do usuário, e pertencem a uma área em constante evolução em ritmo acelerado, como ilustrado no trabalho de \citeonline{9004378}, que destaca que a evolução dos jogos digitais tem sido marcada por avanços exponenciais em Inteligência Artificial (IA), gráficos e portabilidade. Os autores ainda trazem como estudo de caso a \textit{Rockstar Games}, a empresa responsável pelo GTA (\textit{Grand Theft Auto}) -- um jogo que passou de gráficos 2D simples (1997) para mundos abertos com forte ênfase no realismo em 3D (2013) em apenas 16 anos.

De forma sucinta, os jogos são sistemas interativos baseados em regras, que combinam desafios, narrativas e mecânicas para criar experiências imersivas e significativas, seja em ambientes físicos ou digitais. \citeonline{halfreal} e \citeonline{artofgame} elucidam que, independentemente do formato, todo jogo possui regras claras, objetivos definidos e consequências negociáveis, sendo uma ferramenta tanto de entretenimento quanto de aprendizado. Mas, além disso, segundo \citeonline{homoludens}, os jogos também são fenômenos culturais, que representam uma evasão voluntária da realidade.

Assim como os jogos, a Inteligência Artificial tem se tornado cada vez mais presente no cotidiano das pessoas. Segundo \citeonline{iapeter}, a Inteligência Artificial é o estudo de agentes que recebem percepções do ambiente e realizam ações. Os autores classificam os agentes inteligentes como máquinas que pensam como humanos, agem como humanos, pensam racionalmente e agem racionalmente. Por sua vez, a IA generativa refere-se a uma classe de modelos de Inteligência Artificial, baseados em \textit{deep learning}, que podem criar novos conteúdos, como imagens, texto e música \cite{madaan:2024}.  \citeonline{genai} destacam que ferramentas como o \textit{ChatGPT}, \textit{Midjourney} e \textit{DALL-E} tornaram a IA mais acessível para o usuário final, permitindo a criação de conteúdos complexos por meio da interação com um \textit{prompt}. 

Existe uma quantidade considerável de trabalhos relativamente recentes que unem esses dois campos. O trabalho de \citeonline{ref-intro-1}, intitulado ``\textit{AI for Game Production}'' propõe que a IA funcione como um produtor inteligente, atuando na gerência de múltiplos jogos, comunidades e operações em tempo real. \citeonline{chawla24} propõe um \textit{framework} que gera todos os elementos narrativos em tempo real usando modelos GPT (\textit{Generative Pre-trained Transformers}).   \citeonline{ref-intro-2} investiga o impacto da IA generativa no desenvolvimento de jogos independentes através da análise de 3.091 discussões online. 

Dada a versatilidade da IA e sua usabilidade dentro e fora do mundo dos jogos, surge a seguinte questão de pesquisa: é possível utilizar ferramentas de IA generativa para gerar narrativas de jogos coerentes e imersivas? Embora a IA já seja aplicada em jogos para criação de personagens e mundos proceduralmente gerados \cite{chawla24}, a geração de narrativas apresenta desafios específicos, como coesão, adaptação às escolhas do jogador e equilíbrio entre estruturação e liberdade criativa \cite{young2013plans}. 

\citeonline{brom2006petri} propõem o uso de Redes de Petri para representar narrativas não-lineares em jogos. Os autores demonstram que é possível adaptar a narrativa em tempo real com base nas ações dos jogadores, mantendo ao mesmo tempo uma estrutura narrativa coerente. Ainda na mesma linha de pesquisa, \citeonline{brom2010petri} apresentam um método para usar redes de Petri para representar tramas ramificadas em aplicativos de contar histórias, particularmente em jogos sérios. Esse método permite que as histórias evoluam em paralelo, o que é essencial para grandes mundos virtuais.

As redes de Petri permitem a integração de restrições temporais e causais, garantindo que a narrativa se desenvolva de maneira lógica e oportuna. Isso é crucial para manter o engajamento dos jogadores e garantir que a história chegue a uma conclusão satisfatória \cite{8847967}.  Por exemplo, em \citeonline{barreto2015modelagem} a rede de Petri é usada para realizar a modelagem das atividades de um nível de jogo e para a representação do mapa topológico correspondente do mundo virtual do jogo. Assim, o autor conseguiu criar um cenário de jogo que expressasse tanto a narrativa quanto a topologia do mundo virtual, ainda em fases iniciais de criação do jogo. 

Não é novidade o uso das redes de Petri para modelar jogos. Em \citeonline{natkin}, por exemplo, os autores já traziam essa teoria para modelar e analisar as ações de jogos eletrônicos. Assim como \citeonline{digra}, que usou as redes de Petri para modelar e simular o fluxo de jogo. Já em \citeonline{nedopetalski}, os autores apresentam uma forma de traduzir um modelo em rede de Petri para uma ferramenta de modelagem 3D, a fim de representar o cenário de jogo sob uma perspectiva formal e também visual, contribuindo para o processo de game design. 

Portanto, as redes de Petri provaram ser uma ferramenta versátil e poderosa para validação narrativa em jogos, oferecendo uma série de vantagens que as tornam ideais para modelar narrativas complexas. Elas têm a capacidade de lidar com histórias ramificadas, tramas paralelas e restrições temporais, podendo garantir que a narrativa permaneça coerente e envolvente. 

Assim, este trabalho tem como objetivo a geração e validação de narrativas de narrativas de jogos utilizando ferramentas de IA generativa. Para tanto, foi realizado um levantamento das IAs generativas e identificado as três mais populares, gratuitas e com potencial de gerar textos longos. Foram também estudados e definidos critérios  essenciais para boas narrativas. A partir disso, foi estabelecido um roteiro base para alimentar as IAs generativas selecionadas. Cada uma delas gerou um texto correspondente a narrativa que foi analisado levando-se em consideração os elementos que garantem uma boa narrativa. Com isso, obteve-se um modelo de prompt que pode ser usado para gerar narrativas em IA. Além dos critérios, a narrativa também pôde ser analisada, em termos de estrutura lógica, utilizando um modelo em rede de Petri, de acordo com o método de modelagem e análise proposto em \cite{barreto2015modelagem}.

Este trabalho contribui ao explorar o potencial das IAs generativas na criação de narrativas de jogos, demonstrando como um \textit{prompt} bem estruturado, fundamentado em elementos essenciais de boas narrativas, pode orientar esses modelos a gerar textos mais coesos, imersivos e adequados ao contexto de game design. Para isso, este estudo:

\begin{itemize}
\item Desenvolve um roteiro metodológico para orientar a criação de narrativas de jogo com IA generativa, estabelecendo um modelo de \textit{prompt} capaz de ambientar a IA no domínio narrativo e maximizar a qualidade dos resultados produzidos;
\item Propõe e aplica um método de validação formal, baseado em redes de Petri, para analisar a coerência, a estrutura lógica e o fluxo narrativo das histórias geradas, contribuindo para a avaliação de narrativas produzidas por IA;
\item Demonstra, por meio de um exemplo prático, como a IA generativa pode ser integrada ao processo de desenvolvimento de jogos, mais especificamente na etapa de especificação da narrativa, ampliando as possibilidades criativas, acelerando o processo de concepção e fortalecendo a colaboração entre \textit{design} narrativo e ferramentas de IA.
\end{itemize}
