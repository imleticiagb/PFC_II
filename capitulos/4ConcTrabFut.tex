\section{Introdução}
Este capítulo tem como objetivo apresentar os principais pontos discutidos no trabalho, relacionar os possíveis trabalhos futuros advindos desta pesquisa e avaliar a principal contribuição deste trabalho para a área científica.


\section{Conclusões}

\noindent Este trabalho demonstrou a viabilidade da utilização de IA generativa para a criação de narrativas de jogos coerentes e imersivas, integrando conceitos cinematográficos com as particularidades do \textit{design} de jogos. Através da elaboração de um \textit{prompt} estruturado baseado nos princípios de narrativa de três atos, foi possível orientar modelos de linguagem na geração de tramas consistentes que atendem aos critérios fundamentais de uma boa narrativa.

A aplicação da técnica de \textit{role prompting}, posicionando a IA no papel de um \textit{game designer}, mostrou-se eficaz na produção de conteúdos que não apenas seguem uma estrutura narrativa sólida, mas também incorporam elementos essenciais, tais como desenvolvimento de personagens, conflito central e ambientação. A comparação entre diferentes LLMs revelou que o \textit{Deepseek V3} apresentou o melhor desempenho na geração de narrativas com riqueza descritiva e coerência estrutural.

Este trabalho oferece contribuições significativas para a área de desenvolvimento de jogos digitais e Inteligência Artificial, estabelecendo pontes metodológicas entre essas áreas e propondo abordagens inovadoras para desafios criativos e técnicos. A abordagem proposta por esta pesquisa oferece uma metodologia estruturada para o desenvolvimento narrativo, a possibilidade de validação formal das estruturas narrativas e propondo o uso da engenharia de \textit{prompt} especializada para narrativas de jogos, funcionando como uma ponte interdisciplinar.

\section{Trabalhos futuros}

\noindent Com base nos resultados e limitações identificadas neste trabalho, sugere-se as seguintes direções para pesquisas futuras:

\begin{itemize}
    \item \textbf{Expansão para narrativas não-lineares:} desenvolver e testar \textbf{prompts} que orientem a geração de tramas ramificadas, onde as decisões do jogador impactam significativamente o desenrolar da história, utilizando redes de Petri mais complexas para modelar e validar múltiplos caminhos narrativos.
    \item \textbf{Integração com mecânicas de jogo:} explorar a geração conjunta de narrativa e mecânicas de \textbf{gameplay}, investigando como a IA pode criar histórias que se integrem organicamente com os sistemas interativos do jogo.
    \item \textbf{Emprego da metodologia em um ambiente de desenvolvimento real:} conduzir um estudo de caso aplicando a metodologia proposta em um estúdio de desenvolvimento de jogos real.

\end{itemize}
\section{Considerações finais}

Este trabalho demonstrou a viabilidade de uma abordagem metodológica que integra IA generativa e modelagem formal para a criação e validação de narrativas de jogos. Os resultados obtidos indicam que é possível, através de um \textit{prompt} estruturado e especializado, orientar LLMs na geração de tramas coerentes e imersivas que atendem aos critérios fundamentais da teoria narrativa.

Do ponto de vista prático, a metodologia desenvolvida oferece aos desenvolvedores de jogos um roteiro sistemático para a criação de narrativas, potencialmente acelerando as fases iniciais de pré-produção e permitindo a exploração de múltiplas direções criativas. A combinação entre a capacidade generativa da IA e a precisão analítica das redes de Petri representa um avanço significativo na busca por ferramentas que ampliem, sem substituir, a criatividade humana no desenvolvimento de jogos.

No entanto, embora os resultados sejam promissores, reconhece-se que a abordagem proposta se beneficia significativamente do conhecimento prévio do usuário tanto em teoria narrativa quanto no funcionamento de sistemas de IA. A qualidade das saídas está diretamente relacionada com a qualidade e especificidade das instruções fornecidas, reforçando que a IA atua como amplificadora, e não substituta, da \textit{expertise} humana.

Em síntese, este trabalho contribui para o avanço do estado da arte no desenvolvimento de jogos ao demonstrar que a integração entre IA generativa e o desenvolvimento de especificação de jogos. Além disso, os critérios estabelecidos para uma boa narrativa ajudam a avaliar qualitativamente as saídas obtidas. A pesquisa proposta neste trabalho abre caminho para novas pesquisas e aplicações práticas que explorem todo o potencial desta abordagem no cenário cada vez mais complexo e competitivo da indústria de \textit{games}.