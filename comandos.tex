\definecolor{mypink1}{rgb}{0.858, 0.188, 0.478}
\definecolor{mypink2}{RGB}{219, 48, 122}
\definecolor{mypink3}{cmyk}{0, 0.7808, 0.4429, 0.1412}
\definecolor{mygray}{gray}{0.6}

% Comandos para correção
\newcommand{\comentario}[1]{\textcolor{red}{#1}}
\newcommand{\marcos}[1]{\textcolor{blue}{#1}}

% Comandos para Tecnologia e dispositivos
\newcommand{\generico}[0]{GENERICO \texttrademark}


% Comandos para palavras muito usadas no trabalho
\newcommand{\bci}[0]{BCI}
\newcommand{\textbci}[0]{Brain–Computer Interface}

\newcommand{\avc}[0]{CCM}
\newcommand{\textavc}[0]{Como construir uma monografia}




% Altera o nome padrão do rótulo usado no comando \autoref{}
\renewcommand{\lstlistingname}{Código}
\renewcommand{\appendixautorefname}{Anexo} %%%% ===> aqui o teste
\renewcommand{\tableautorefname}{Tabela} 
% Altera o rótulo a ser usando no elemento pré-textual "Lista de código"
\renewcommand{\lstlistlistingname}{Lista de códigos}





% Configura a ``Lista de Códigos'' conforme as regras da ABNT (para abnTeX2)
\begingroup\makeatletter
\let\newcounter\@gobble\let\setcounter\@gobbletwo
  \globaldefs\@ne \let\c@loldepth\@ne
  \newlistof{listings}{lol}{\lstlistlistingname}
  \newlistentry{lstlisting}{lol}{0}
\endgroup

\renewcommand{\cftlstlistingaftersnum}{\hfill--\hfill}

\let\oldlstlistoflistings\lstlistoflistings
\renewcommand{\lstlistoflistings}{%
   \begingroup%
   \let\oldnumberline\numberline%
   \renewcommand{\numberline}{\lstlistingname\space\oldnumberline}%
   \oldlstlistoflistings%
   \endgroup}

\definecolor{dkgreen}{rgb}{0,0.6,0}
\definecolor{gray}{rgb}{0.5,0.5,0.5}
\definecolor{mauve}{rgb}{0.58,0,0.82}
\definecolor{light-gray}{gray}{0.25}

\lstdefinestyle{java}{
  language=Java,
  aboveskip=3mm,
  belowskip=3mm,
  showstringspaces=false,
  columns=flexible,
  basicstyle={\footnotesize\ttfamily},
  numberstyle={\tiny},
  numbers=left,
  keywordstyle=\color{blue},
  commentstyle=\color{dkgreen},
  stringstyle=\color{mauve},
  breaklines=true,
  breakatwhitespace=true,
  tabsize=3
}

\lstdefinestyle{bytecode}{
  otherkeywords={invokedynamic},
  language=JVMIS,
  aboveskip=3mm,
  belowskip=3mm,
  showstringspaces=false,
  columns=flexible,
  basicstyle={\footnotesize\ttfamily},
  numberstyle={\tiny},
  numbers=left,
  keywordstyle=\color{blue},
  commentstyle=\color{dkgreen},
  stringstyle=\color{mauve},
  breaklines=true,
  breakatwhitespace=true,
  tabsize=3
}

\lstdefinestyle{fsharp} {	
  morekeywords={let, new, match, with, rec, 
    open, module, namespace, type, of, member, 
    and, for, while, true, false, in, do, begin, 
    end, fun, function, return, yield, try, val, 
    mutable, if, then, else, cloud, async, static, 
    use, abstract, interface, inherit, finally, maybe, option },
  otherkeywords={ let!, return!, do!, yield!, use!, var, from, select, where, order},
  keywordstyle=\color{blue},
  sensitive=true,
  aboveskip=3mm,
  belowskip=3mm,
  showstringspaces=false, 
  columns=flexible,
  basicstyle={\footnotesize\ttfamily},
  numberstyle={\tiny},
  numbers=left,
  breaklines=true,
  upquote=true,
  tabsize=3,
  morecomment=[l][\color{dkgreen}]{///},
  morecomment=[l][\color{dkgreen}]{//},
  morecomment=[s][\color{dkgreen}]{{(*}{*)}},
  morestring=[b]",
  showstringspaces=false,
  literate={`}{\`}1,
  stringstyle=\color{mauve}
}

\lstdefinestyle{csharp} {
  aboveskip=3mm,
  belowskip=3mm,
  showstringspaces=false,
  columns=flexible,
  basicstyle={\footnotesize\ttfamily},
  numberstyle={\tiny},
  numbers=left,
  keywordstyle=\color{blue},
  commentstyle=\color{dkgreen},
  stringstyle=\color{mauve},
  breaklines=true,
  breakatwhitespace=true,
  tabsize=3,
  morecomment = [l]{//}, 
  morecomment = [l]{///},
  morecomment = [s]{/*}{*/},
  morestring=[b]", 
  sensitive = true,
  morekeywords = {async, await, abstract,  
    event,  new,  struct,
    as,  explicit,  null,  switch,
    base,  extern,  object,  this,
    bool,  false,  operator,  throw,
    break,  finally,  out,  true,
    byte,  fixed,  override,  try,
    case,  float,  params,  typeof,
    catch,  for,  private,  uint,
    char,  foreach,  protected,  ulong,
    checked,  goto,  public,  unchecked,
    class,  if,  readonly,  unsafe,
    const,  implicit,  ref,  ushort,
    continue,  in,  return,  using,
    decimal,  int,  sbyte,  virtual,
    default,  interface,  sealed,  volatile,
    delegate,  internal,  short,  void,
    do,  is,  sizeof,  while,
    double,  lock,  stackalloc,   
    else,  long,  static,   
    enum,  namespace,  string }
}

\lstdefinestyle{scala} {  
  morekeywords={ abstract,case,catch,
    char,class,
    def,else,extends,final,
    if,import,
    match,module,new,null,object,
    override,package,private,protected,
    public,return,super,this,throw,
    trait,try,type,val,var,with,implicit,
    macro,sealed
  },
  sensitive,
  morecomment=[l]//,
  morecomment=[s]{/*}{*/},
  morestring=[b]",
  morestring=[b]',
  aboveskip=3mm,
  belowskip=3mm,
  showstringspaces=false,
  columns=flexible,
  basicstyle={\footnotesize\ttfamily},
  numberstyle={\tiny},
  numbers=left,
  keywordstyle=\color{blue},
  commentstyle=\color{dkgreen},
  stringstyle=\color{mauve},
  breaklines=true,
  breakatwhitespace=true,
  tabsize=3
}
\lstset{escapechar=@,style=java}

% ---
% Pacotes de citações
% ---
\usepackage[brazilian,hyperpageref]{backref}	 
% Paginas com as citações na bibl
\usepackage[alf]{abntex2cite}	
% Citações padrão ABNT

\usepackage{pdfpages}
% ---
% Configurações do pacote backref
% Usado sem a opção hyperpageref de backref
\renewcommand{\backrefpagesname}{Citado na(s) página(s):~}
% Texto padrão antes do número das páginas
\renewcommand{\backref}{}
% Define os textos da citação
\renewcommand*{\backrefalt}[4]{
	\ifcase #1 %
		Nenhuma citação no texto.%
	\or
		Citado na página #2.%
	\else
		Citado #1 vezes nas páginas #2.%
	\fi}%
% ---


% alterando o aspecto da cor azul
\definecolor{blue}{RGB}{41,5,195}

% informações do PDF
\makeatletter
\hypersetup{
     	%pagebackref=true,
		pdftitle={\@title}, 
		pdfauthor={\@author},
    	pdfsubject={\imprimirpreambulo},
	    pdfcreator={LaTeX with abnTeX2},
		pdfkeywords={abnt}{latex}{abntex}{abntex2}{trabalho acadêmico}, 
		colorlinks=true,       		% false: boxed links; true: colored links
    	linkcolor=blue,          	% color of internal links
    	citecolor=blue,        		% color of links to bibliography
    	filecolor=magenta,      		% color of file links
		urlcolor=blue,
		bookmarksdepth=4
}
\makeatother
% --- 

% --- 
% Espaçamentos entre linhas e parágrafos 
% --- 
% O tamanho do parágrafo é dado por:
\setlength{\parindent}{1.3cm}
% Controle do espaçamento entre um parágrafo e outro:
\setlength{\parskip}{0.2cm}  % tente também \onelineskip
% ---
% compila o indice
% ---
\makeindex
% ---
