%% abtex2-modelo-projeto-pesquisa.tex, v-1 marcoswagner PFC 2 2023
%% Copyright 2020-2023 by abnTeX2 group at BCC-UFJ http://www.abntex.net.br/ 
%%
%% This work consists of the files abntex2-modelo-projeto-pesquisa.tex
%% and abntex2-modelo-references.bib
%%

% ---------------------------------------------------------------------
% ---------------------------------------------------------------------
% abnTeX2: Modelo Adaptado de Monografia em conformidade com 
% ABNT NBR 15287:2011 Informação e documentação - Monografia 
% --------------------------------------------------------------------- 
% ---------------------------------------------------------------------
\documentclass[
	12pt,				
	oneside,
	a4paper,		
    sumario=tradicional,
	english,			
	french,				
	spanish,			
	brazil,				
	documento
	]{abntex2}

% ---
% Configurações
% ---
\usepackage{abntex2/abntex2-bcc-ufj}
\renewcommand*\arraystretch{1.2} 
\usepackage{pdfpages} %para incluir pdf como páginas


% ---
% Pacotes básicos 
% ---
\usepackage{lmodern}					
\usepackage[T1]{fontenc}		
\usepackage[utf8]{inputenc}	
\usepackage{lastpage}		
\usepackage{indentfirst}	
\usepackage{color}			
\usepackage{graphicx}		
\usepackage{microtype} 		
% ---

% ---
% Pacotes adicionais,
\usepackage{lipsum}	
\usepackage{caption}
\usepackage{subcaption}
\usepackage{enumerate} 
\usepackage{listings}     


% --- 
% CONFIGURAÇÕES DE PACOTES
% --- 
\definecolor{mypink1}{rgb}{0.858, 0.188, 0.478}
\definecolor{mypink2}{RGB}{219, 48, 122}
\definecolor{mypink3}{cmyk}{0, 0.7808, 0.4429, 0.1412}
\definecolor{mygray}{gray}{0.6}

% Comandos para correção
\newcommand{\comentario}[1]{\textcolor{red}{#1}}
\newcommand{\marcos}[1]{\textcolor{blue}{#1}}

% Comandos para Tecnologia e dispositivos
\newcommand{\generico}[0]{GENERICO \texttrademark}


% Comandos para palavras muito usadas no trabalho
\newcommand{\bci}[0]{BCI}
\newcommand{\textbci}[0]{Brain–Computer Interface}

\newcommand{\avc}[0]{CCM}
\newcommand{\textavc}[0]{Como construir uma monografia}




% Altera o nome padrão do rótulo usado no comando \autoref{}
\renewcommand{\lstlistingname}{Código}
\renewcommand{\appendixautorefname}{Anexo} %%%% ===> aqui o teste
\renewcommand{\tableautorefname}{Tabela} 
% Altera o rótulo a ser usando no elemento pré-textual "Lista de código"
\renewcommand{\lstlistlistingname}{Lista de códigos}





% Configura a ``Lista de Códigos'' conforme as regras da ABNT (para abnTeX2)
\begingroup\makeatletter
\let\newcounter\@gobble\let\setcounter\@gobbletwo
  \globaldefs\@ne \let\c@loldepth\@ne
  \newlistof{listings}{lol}{\lstlistlistingname}
  \newlistentry{lstlisting}{lol}{0}
\endgroup

\renewcommand{\cftlstlistingaftersnum}{\hfill--\hfill}

\let\oldlstlistoflistings\lstlistoflistings
\renewcommand{\lstlistoflistings}{%
   \begingroup%
   \let\oldnumberline\numberline%
   \renewcommand{\numberline}{\lstlistingname\space\oldnumberline}%
   \oldlstlistoflistings%
   \endgroup}

\definecolor{dkgreen}{rgb}{0,0.6,0}
\definecolor{gray}{rgb}{0.5,0.5,0.5}
\definecolor{mauve}{rgb}{0.58,0,0.82}
\definecolor{light-gray}{gray}{0.25}

\lstdefinestyle{java}{
  language=Java,
  aboveskip=3mm,
  belowskip=3mm,
  showstringspaces=false,
  columns=flexible,
  basicstyle={\footnotesize\ttfamily},
  numberstyle={\tiny},
  numbers=left,
  keywordstyle=\color{blue},
  commentstyle=\color{dkgreen},
  stringstyle=\color{mauve},
  breaklines=true,
  breakatwhitespace=true,
  tabsize=3
}

\lstdefinestyle{bytecode}{
  otherkeywords={invokedynamic},
  language=JVMIS,
  aboveskip=3mm,
  belowskip=3mm,
  showstringspaces=false,
  columns=flexible,
  basicstyle={\footnotesize\ttfamily},
  numberstyle={\tiny},
  numbers=left,
  keywordstyle=\color{blue},
  commentstyle=\color{dkgreen},
  stringstyle=\color{mauve},
  breaklines=true,
  breakatwhitespace=true,
  tabsize=3
}

\lstdefinestyle{fsharp} {	
  morekeywords={let, new, match, with, rec, 
    open, module, namespace, type, of, member, 
    and, for, while, true, false, in, do, begin, 
    end, fun, function, return, yield, try, val, 
    mutable, if, then, else, cloud, async, static, 
    use, abstract, interface, inherit, finally, maybe, option },
  otherkeywords={ let!, return!, do!, yield!, use!, var, from, select, where, order},
  keywordstyle=\color{blue},
  sensitive=true,
  aboveskip=3mm,
  belowskip=3mm,
  showstringspaces=false, 
  columns=flexible,
  basicstyle={\footnotesize\ttfamily},
  numberstyle={\tiny},
  numbers=left,
  breaklines=true,
  upquote=true,
  tabsize=3,
  morecomment=[l][\color{dkgreen}]{///},
  morecomment=[l][\color{dkgreen}]{//},
  morecomment=[s][\color{dkgreen}]{{(*}{*)}},
  morestring=[b]",
  showstringspaces=false,
  literate={`}{\`}1,
  stringstyle=\color{mauve}
}

\lstdefinestyle{csharp} {
  aboveskip=3mm,
  belowskip=3mm,
  showstringspaces=false,
  columns=flexible,
  basicstyle={\footnotesize\ttfamily},
  numberstyle={\tiny},
  numbers=left,
  keywordstyle=\color{blue},
  commentstyle=\color{dkgreen},
  stringstyle=\color{mauve},
  breaklines=true,
  breakatwhitespace=true,
  tabsize=3,
  morecomment = [l]{//}, 
  morecomment = [l]{///},
  morecomment = [s]{/*}{*/},
  morestring=[b]", 
  sensitive = true,
  morekeywords = {async, await, abstract,  
    event,  new,  struct,
    as,  explicit,  null,  switch,
    base,  extern,  object,  this,
    bool,  false,  operator,  throw,
    break,  finally,  out,  true,
    byte,  fixed,  override,  try,
    case,  float,  params,  typeof,
    catch,  for,  private,  uint,
    char,  foreach,  protected,  ulong,
    checked,  goto,  public,  unchecked,
    class,  if,  readonly,  unsafe,
    const,  implicit,  ref,  ushort,
    continue,  in,  return,  using,
    decimal,  int,  sbyte,  virtual,
    default,  interface,  sealed,  volatile,
    delegate,  internal,  short,  void,
    do,  is,  sizeof,  while,
    double,  lock,  stackalloc,   
    else,  long,  static,   
    enum,  namespace,  string }
}

\lstdefinestyle{scala} {  
  morekeywords={ abstract,case,catch,
    char,class,
    def,else,extends,final,
    if,import,
    match,module,new,null,object,
    override,package,private,protected,
    public,return,super,this,throw,
    trait,try,type,val,var,with,implicit,
    macro,sealed
  },
  sensitive,
  morecomment=[l]//,
  morecomment=[s]{/*}{*/},
  morestring=[b]",
  morestring=[b]',
  aboveskip=3mm,
  belowskip=3mm,
  showstringspaces=false,
  columns=flexible,
  basicstyle={\footnotesize\ttfamily},
  numberstyle={\tiny},
  numbers=left,
  keywordstyle=\color{blue},
  commentstyle=\color{dkgreen},
  stringstyle=\color{mauve},
  breaklines=true,
  breakatwhitespace=true,
  tabsize=3
}
\lstset{escapechar=@,style=java}

% ---
% Pacotes de citações
% ---
\usepackage[brazilian,hyperpageref]{backref}	 
% Paginas com as citações na bibl
\usepackage[alf]{abntex2cite}	
% Citações padrão ABNT

\usepackage{pdfpages}
% ---
% Configurações do pacote backref
% Usado sem a opção hyperpageref de backref
\renewcommand{\backrefpagesname}{Citado na(s) página(s):~}
% Texto padrão antes do número das páginas
\renewcommand{\backref}{}
% Define os textos da citação
\renewcommand*{\backrefalt}[4]{
	\ifcase #1 %
		Nenhuma citação no texto.%
	\or
		Citado na página #2.%
	\else
		Citado #1 vezes nas páginas #2.%
	\fi}%
% ---


% alterando o aspecto da cor azul
\definecolor{blue}{RGB}{41,5,195}

% informações do PDF
\makeatletter
\hypersetup{
     	%pagebackref=true,
		pdftitle={\@title}, 
		pdfauthor={\@author},
    	pdfsubject={\imprimirpreambulo},
	    pdfcreator={LaTeX with abnTeX2},
		pdfkeywords={abnt}{latex}{abntex}{abntex2}{trabalho acadêmico}, 
		colorlinks=true,       		% false: boxed links; true: colored links
    	linkcolor=blue,          	% color of internal links
    	citecolor=blue,        		% color of links to bibliography
    	filecolor=magenta,      		% color of file links
		urlcolor=blue,
		bookmarksdepth=4
}
\makeatother
% --- 

% --- 
% Espaçamentos entre linhas e parágrafos 
% --- 
% O tamanho do parágrafo é dado por:
\setlength{\parindent}{1.3cm}
% Controle do espaçamento entre um parágrafo e outro:
\setlength{\parskip}{0.2cm}  % tente também \onelineskip
% ---
% compila o indice
% ---
\makeindex
% ---


% ---
% Informações de dados para CAPA e FOLHA DE ROSTO
% ---
% ----------------------------------------------------------
% DADOS DO TRABALHO - CAPA e FOLHA DE ROSTO
% Preencha todos os dados aqui...
% ----------------------------------------------------------
\titulo{Inteligência Artificial como Autora: Geração e Validação Formal de Narrativas para Jogos}
\autor{Leticia Gonçalves Batista}
\local{Jataí - Goiás}
\data{Novembro de 2025}
\orientador{Prof.ª Dr.ª Franciny Medeiros Barreto}
\coorientador{}
\instituicao{%
  UNIVERSIDADE FEDERAL DE JATAÍ (UFJ)
  \par
  INSTITUTO DE CIÊNCIAS EXATAS E TECNOLÓGICAS (ICET)
  \par
  CURSO DE CIÊNCIA DA COMPUTAÇÃO}
\tipotrabalho{Monografia (Graduação)}
% O preambulo deve conter o tipo do trabalho, o objetivo, 
% o nome da instituição e a área de concentração 
\preambulo{Monografia apresentada ao curso de   Ciência da Computação do Instituto de Ciências Exatas e Tecnológicas da Universidade Federal de Jataí (UFJ), como requisito para obtenção do título de Bacharel em Ciência da Computação.}



% ---




%%%%%%%%%=============== DOCUMENTO ===========%%%%%%%%%%%
% ----
% Início do documento
% ----
\begin{document}
\selectlanguage{brazil}
% Retira espaço extra obsoleto entre as frases.
\frenchspacing 

% ----------------------------------------------------------
% ELEMENTOS PRÉ-TEXTUAIS
% ----------------------------------------------------------
% \pretextual



% ---
 \begin{fichacatalografica} 
 \includepdf[pages={-}]{FichaCatalografica.pdf} 
 \end{fichacatalografica}



% ---
 \includepdf{aprova}
% ---

% ---
% RESUMOS
% ---

% resumo em português
\setlength{\absparsep}{18pt} % ajusta o espaçamento dos parágrafos do resumo
\begin{resumo}

Este trabalho investiga o uso de Inteligência Artificial generativa como ferramenta para criação de narrativas de jogos digitais, examinando se modelos de linguagem de grande escala (LLMs) são capazes de produzir histórias coerentes, cinematográficas e adequadas às necessidades do \textit{game design}. O problema central reside na dificuldade de gerar narrativas que mantenham coesão interna, progressão dramática e lógica causals. O objetivo principal é propor e avaliar um método que permita tanto a geração quanto a validação formal dessas narrativas, estabelecendo critérios estruturais baseados nos estudos de cinema e nos princípios narrativos aplicados ao desenvolvimento de jogos. A pesquisa justifica-se pela crescente presença da IA no processo criativo, pela demanda por métodos acessíveis para desenvolvedores independentes e pela escassez de estudos que integrem técnicas narrativas, modelos generativos e ferramentas formais de validação. Metodologicamente, este trabalho parte da definição de elementos essenciais de uma boa narrativa e da construção de um \textit{prompt} detalhado baseado em personificação (\textit{role prompting}). Esse \textit{prompt} foi aplicado em três modelos de IA amplamente utilizados, sendo eles: \textit{GPT-5}, \textit{DeepSeek V3} e \textit{Gemini 2.5 Flash}. O objetivo era gerar narrativas comparáveis. As produções foram analisadas a partir de critérios formais estabelecidos na literatura e, posteriormente, traduzidas para um modelo de rede de Petri, a fim de verificar propriedades como coerência estrutural, fluxo lógico e progressão causal. Os resultados demonstram que a IA é capaz de produzir narrativas consistentes quando orientada por um \textit{prompt} bem estruturado, sendo o modelo \textit{DeepSeek V3} o que apresentou melhor aderência aos critérios definidos. A modelagem em rede de Petri confirmou a possibilidade de validação formal de narrativas geradas automaticamente, evidenciando sua estrutura lógica. Como contribuição, esta pesquisa propõe um roteiro metodológico replicável e acessível, que integra engenharia de \textit{prompt} e validação formal, oferecendo uma abordagem inédita para o uso de IA generativa na concepção narrativa de jogos.

 \textbf{Palavras-chaves}: \textit{Inteligência Artificial Generativa; Narrativas para Jogos; Redes de Petri; Modelagem; Game Design.} 


\end{resumo}

% resumo em inglês
\begin{resumo}[Abstract]
 \begin{otherlanguage*}{english}
 This work investigates the use of generative Artificial Intelligence as a tool for the creation of digital game narratives, examining whether large language models (LLMs) are capable of producing coherent, cinematic stories that fit the needs of game design. The central problem lies in the difficulty of generating narratives that maintain internal cohesion, dramatic progression, and causal logic. The main objective is to propose and evaluate a method that allows both the generation and formal validation of these narratives, establishing structural criteria based on film studies and narrative principles applied to game development. The research is justified by the growing presence of AI in the creative process, the demand for accessible methods for independent developers, and the scarcity of studies that integrate narrative techniques, generative models, and formal validation tools. Methodologically, this work begins with the definition of essential elements of a good narrative and the construction of a detailed prompt based on personification (role prompting). This prompt was applied to three widely used AI models: GPT-5, DeepSeek V3, and Gemini 2.5 Flash. The objective was to generate comparable narratives. The outputs were analyzed using formal criteria established in the literature and later translated into a Petri net model in order to verify properties such as structural coherence, logical flow, and causal progression. The results demonstrate that AI is capable of producing consistent narratives when guided by a well-structured prompt, with the DeepSeek V3 model showing the best adherence to the defined criteria. The Petri net modeling confirmed the possibility of formal validation of automatically generated narratives, highlighting their logical structure. As a contribution, this research proposes a replicable and accessible methodological framework that integrates prompt engineering and formal validation, offering a novel approach to the use of generative AI in narrative design for games.
   
   \textbf{Key-words}: 
   \textit{Generative Artificial Intelligence; Game Narratives; Petri Nets; Modeling; Game Design.}
 \end{otherlanguage*}
\end{resumo}



% ---
% Dedicatória/Agradecimentos/Epígrafe
\begin{dedicatoria}
   \vspace*{\fill}
   \centering
   \noindent
   \textit{
  Dedico este trabalho à minha família, que sempre acreditou em mim e me apoiou em cada passo dessa caminhada. À minha orientadora, pela inspiração e por me mostrar que o conhecimento vai muito além das páginas. E dedico também a todos que estiveram ao meu lado nos dias difíceis, me lembrando de que eu era capaz. Este trabalho é para vocês, que fizeram parte da minha força e da minha motivação para chegar até aqui.
   } \vspace*{\fill}
\end{dedicatoria}
% ---

% ---
% Agradecimentos
% ---
\begin{agradecimentos}

 \vspace*{\fill}
   \centering
   \noindent
   \textit{
Agradeço, primeiramente, à minha orientadora, pela paciência, dedicação e pela orientação maravilhosa que guiou cada etapa deste trabalho. Sua confiança, disponibilidade e olhar cuidadoso foram fundamentais para meu crescimento acadêmico e pessoal ao longo desta jornada. À minha família, expresso minha profunda gratidão pelo amor, compreensão e apoio incondicional. Cada palavra de incentivo e cada gesto de carinho me deram forças para continuar, mesmo nos momentos mais desafiadores. Aos meus amigos, deixo meu sincero agradecimento pelo suporte emocional e por sempre estarem ao meu lado quando eu achava que não ia conseguir. A presença, o companheirismo e as conversas que me acalmaram fizeram toda a diferença. Agradeço também aos membros da banca examinadora, pela disponibilidade, pelas contribuições valiosas e pelas considerações que ajudaram a aperfeiçoar este trabalho. Suas observações ampliaram minha visão e enriqueceram ainda mais este projeto. Por fim, não posso deixar de registrar o quanto estou feliz e realizada por, enfim, ter concluído esta etapa tão importante da minha vida. Este trabalho representa esforço, aprendizado e superação, e sou imensamente grata a todos que fizeram parte desse percurso.
   } \vspace*{\fill}

\end{agradecimentos}
% ---

% ---
% Epígrafe
% ---
\begin{epigrafe}
    \vspace*{\fill}
	\begin{flushright}
		\textit{``Se o conhecimento fosse perigoso, a solução seria a ignorância. Sempre me pareceu que a solução teria que ser a sabedoria.\\
		(Eu Robô, de Isaac Asimov)}
	\end{flushright}
\end{epigrafe}
% ---
% ---
\begin{dedicatoria}
   \vspace*{\fill}
   \centering
   \noindent
   \textit{
  Dedico este trabalho à minha família, que sempre acreditou em mim e me apoiou em cada passo dessa caminhada. À minha orientadora, pela inspiração e por me mostrar que o conhecimento vai muito além das páginas. E dedico também a todos que estiveram ao meu lado nos dias difíceis, me lembrando de que eu era capaz. Este trabalho é para vocês, que fizeram parte da minha força e da minha motivação para chegar até aqui.
   } \vspace*{\fill}
\end{dedicatoria}
% ---

% ---
% Agradecimentos
% ---
\begin{agradecimentos}

 \vspace*{\fill}
   \centering
   \noindent
   \textit{
Agradeço, primeiramente, à minha orientadora, pela paciência, dedicação e pela orientação maravilhosa que guiou cada etapa deste trabalho. Sua confiança, disponibilidade e olhar cuidadoso foram fundamentais para meu crescimento acadêmico e pessoal ao longo desta jornada. À minha família, expresso minha profunda gratidão pelo amor, compreensão e apoio incondicional. Cada palavra de incentivo e cada gesto de carinho me deram forças para continuar, mesmo nos momentos mais desafiadores. Aos meus amigos, deixo meu sincero agradecimento pelo suporte emocional e por sempre estarem ao meu lado quando eu achava que não ia conseguir. A presença, o companheirismo e as conversas que me acalmaram fizeram toda a diferença. Agradeço também aos membros da banca examinadora, pela disponibilidade, pelas contribuições valiosas e pelas considerações que ajudaram a aperfeiçoar este trabalho. Suas observações ampliaram minha visão e enriqueceram ainda mais este projeto. Por fim, não posso deixar de registrar o quanto estou feliz e realizada por, enfim, ter concluído esta etapa tão importante da minha vida. Este trabalho representa esforço, aprendizado e superação, e sou imensamente grata a todos que fizeram parte desse percurso.
   } \vspace*{\fill}

\end{agradecimentos}
% ---

% ---
% Epígrafe
% ---
\begin{epigrafe}
    \vspace*{\fill}
	\begin{flushright}
		\textit{``Se o conhecimento fosse perigoso, a solução seria a ignorância. Sempre me pareceu que a solução teria que ser a sabedoria.\\
		(Eu Robô, de Isaac Asimov)}
	\end{flushright}
\end{epigrafe}
% ---




% ---
% inserir lista de ilustrações
% ---
\pdfbookmark[0]{\listfigurename}{lof}
\listoffigures*
\cleardoublepage
% ---


% ---
% inserir o sumario
% ---
\pdfbookmark[0]{\contentsname}{toc}
\tableofcontents*
\cleardoublepage
% ---



% ===================================================
% ELEMENTOS TEXTUAIS - Capítulos ----
% ----------------------------------------------------------
\textual

\chapter[Introdução]{Introdução}\label{cap:introducao}
\addcontentsline{toc}{chapter}{INTRODUÇÃO}
\noindent Definir o mundo do jogo não é uma tarefa trivial, uma vez que os jogos são projetos multidisciplinares que envolvem narrativa, mecânicas, arte, programação e design de experiência do usuário, e pertencem a uma área em constante evolução em ritmo acelerado, como ilustrado no trabalho de \citeonline{9004378}, que destaca que a evolução dos jogos digitais tem sido marcada por avanços exponenciais em Inteligência Artificial (IA), gráficos e portabilidade. Os autores ainda trazem como estudo de caso a \textit{Rockstar Games}, a empresa responsável pelo GTA (\textit{Grand Theft Auto}) -- um jogo que passou de gráficos 2D simples (1997) para mundos abertos com forte ênfase no realismo em 3D (2013) em apenas 16 anos.

De forma sucinta, os jogos são sistemas interativos baseados em regras, que combinam desafios, narrativas e mecânicas para criar experiências imersivas e significativas, seja em ambientes físicos ou digitais. \citeonline{halfreal} e \citeonline{artofgame} elucidam que, independentemente do formato, todo jogo possui regras claras, objetivos definidos e consequências negociáveis, sendo uma ferramenta tanto de entretenimento quanto de aprendizado. Mas, além disso, segundo \citeonline{homoludens}, os jogos também são fenômenos culturais, que representam uma evasão voluntária da realidade.

Assim como os jogos, a Inteligência Artificial tem se tornado cada vez mais presente no cotidiano das pessoas. Segundo \citeonline{iapeter}, a Inteligência Artificial é o estudo de agentes que recebem percepções do ambiente e realizam ações. Os autores classificam os agentes inteligentes como máquinas que pensam como humanos, agem como humanos, pensam racionalmente e agem racionalmente. Por sua vez, a IA generativa refere-se a uma classe de modelos de Inteligência Artificial, baseados em \textit{deep learning}, que podem criar novos conteúdos, como imagens, texto e música \cite{madaan:2024}.  \citeonline{genai} destacam que ferramentas como o \textit{ChatGPT}, \textit{Midjourney} e \textit{DALL-E} tornaram a IA mais acessível para o usuário final, permitindo a criação de conteúdos complexos por meio da interação com um \textit{prompt}. 

Existe uma quantidade considerável de trabalhos relativamente recentes que unem esses dois campos. O trabalho de \citeonline{ref-intro-1}, intitulado ``\textit{AI for Game Production}'' propõe que a IA funcione como um produtor inteligente, atuando na gerência de múltiplos jogos, comunidades e operações em tempo real. \citeonline{chawla24} propõe um \textit{framework} que gera todos os elementos narrativos em tempo real usando modelos GPT (\textit{Generative Pre-trained Transformers}).   \citeonline{ref-intro-2} investiga o impacto da IA generativa no desenvolvimento de jogos independentes através da análise de 3.091 discussões online. 

Dada a versatilidade da IA e sua usabilidade dentro e fora do mundo dos jogos, surge a seguinte questão de pesquisa: é possível utilizar ferramentas de IA generativa para gerar narrativas de jogos coerentes e imersivas? Embora a IA já seja aplicada em jogos para criação de personagens e mundos proceduralmente gerados \cite{chawla24}, a geração de narrativas apresenta desafios específicos, como coesão, adaptação às escolhas do jogador e equilíbrio entre estruturação e liberdade criativa \cite{young2013plans}. 

\citeonline{brom2006petri} propõem o uso de Redes de Petri para representar narrativas não-lineares em jogos. Os autores demonstram que é possível adaptar a narrativa em tempo real com base nas ações dos jogadores, mantendo ao mesmo tempo uma estrutura narrativa coerente. Ainda na mesma linha de pesquisa, \citeonline{brom2010petri} apresentam um método para usar redes de Petri para representar tramas ramificadas em aplicativos de contar histórias, particularmente em jogos sérios. Esse método permite que as histórias evoluam em paralelo, o que é essencial para grandes mundos virtuais.

As redes de Petri permitem a integração de restrições temporais e causais, garantindo que a narrativa se desenvolva de maneira lógica e oportuna. Isso é crucial para manter o engajamento dos jogadores e garantir que a história chegue a uma conclusão satisfatória \cite{8847967}.  Por exemplo, em \citeonline{barreto2015modelagem} a rede de Petri é usada para realizar a modelagem das atividades de um nível de jogo e para a representação do mapa topológico correspondente do mundo virtual do jogo. Assim, o autor conseguiu criar um cenário de jogo que expressasse tanto a narrativa quanto a topologia do mundo virtual, ainda em fases iniciais de criação do jogo. 

Não é novidade o uso das redes de Petri para modelar jogos. Em \citeonline{natkin}, por exemplo, os autores já traziam essa teoria para modelar e analisar as ações de jogos eletrônicos. Assim como \citeonline{digra}, que usou as redes de Petri para modelar e simular o fluxo de jogo. Já em \citeonline{nedopetalski}, os autores apresentam uma forma de traduzir um modelo em rede de Petri para uma ferramenta de modelagem 3D, a fim de representar o cenário de jogo sob uma perspectiva formal e também visual, contribuindo para o processo de game design. 

Portanto, as redes de Petri provaram ser uma ferramenta versátil e poderosa para validação narrativa em jogos, oferecendo uma série de vantagens que as tornam ideais para modelar narrativas complexas. Elas têm a capacidade de lidar com histórias ramificadas, tramas paralelas e restrições temporais, podendo garantir que a narrativa permaneça coerente e envolvente. 

Assim, este trabalho tem como objetivo a geração e validação de narrativas de narrativas de jogos utilizando ferramentas de IA generativa. Para tanto, foi realizado um levantamento das IAs generativas e identificado as três mais populares, gratuitas e com potencial de gerar textos longos. Foram também estudados e definidos critérios  essenciais para boas narrativas. A partir disso, foi estabelecido um roteiro base para alimentar as IAs generativas selecionadas. Cada uma delas gerou um texto correspondente a narrativa que foi analisado levando-se em consideração os elementos que garantem uma boa narrativa. Com isso, obteve-se um modelo de prompt que pode ser usado para gerar narrativas em IA. Além dos critérios, a narrativa também pôde ser analisada, em termos de estrutura lógica, utilizando um modelo em rede de Petri, de acordo com o método de modelagem e análise proposto em \cite{barreto2015modelagem}.

Este trabalho contribui ao explorar o potencial das IAs generativas na criação de narrativas de jogos, demonstrando como um \textit{prompt} bem estruturado, fundamentado em elementos essenciais de boas narrativas, pode orientar esses modelos a gerar textos mais coesos, imersivos e adequados ao contexto de game design. Para isso, este estudo:

\begin{itemize}
\item Desenvolve um roteiro metodológico para orientar a criação de narrativas de jogo com IA generativa, estabelecendo um modelo de \textit{prompt} capaz de ambientar a IA no domínio narrativo e maximizar a qualidade dos resultados produzidos;
\item Propõe e aplica um método de validação formal, baseado em redes de Petri, para analisar a coerência, a estrutura lógica e o fluxo narrativo das histórias geradas, contribuindo para a avaliação de narrativas produzidas por IA;
\item Demonstra, por meio de um exemplo prático, como a IA generativa pode ser integrada ao processo de desenvolvimento de jogos, mais especificamente na etapa de especificação da narrativa, ampliando as possibilidades criativas, acelerando o processo de concepção e fortalecendo a colaboração entre \textit{design} narrativo e ferramentas de IA.
\end{itemize}


\chapter{Referencial Teórico}\label{cap:referencial}

\noindent Neste capítulo, serão fornecidos os conceitos principais para o entendimento dos objetivos da pesquisa descrita neste documento. Na seção \ref{jogos digitais} serão discutidos os jogos eletrônicos, abrangendo definição, tipos e estrutura narrativa. A seção \ref{AI}  abordará os fundamentos da Inteligência Artificial, ao passo que a subseção \ref{gen AI} será dedicada especificamente à Inteligência Artificial Generativa. A seção \ref{modelagem} tratará sobre modelagem de \textit{Software} e, em particular, será apresentada a rede de Petri. 

\section{Jogos Digitais}
\label{jogos digitais}

\noindent Definir o que é um jogo não é uma tarefa simples. Por se tratar de uma área em constante e acelerada evolução, torna-se cada vez mais complexo estabelecer uma definição precisa e abrangente o suficiente para englobar os novos tipos de \textit{games} que surgem ano após ano. Para os propósitos deste trabalho, a definição de \citeonline{gamedesign}, que define jogo como uma forma de entretenimento estruturada e regrada, na qual se tem um objetivo a ser alcançado, será adotada.

\citeonline{homoludens} trata o jogo como um fenômeno cultural, uma evasão voluntária e consciente da vida real, que acontece dentro de um tempo e espaço definido e limitado.  Tanto \citeonline{gamedesign} quanto \citeonline{homoludens} consideram que os jogos têm a capacidade de ensinar, e que todo jogo traz algum tipo de conhecimento. Além disso, \citeonline{homoludens} ainda destaca a importância das regras para o bom funcionamento dos jogos. 

\begin{citacao}
    Regras de jogo são paradoxais: regras e diversão podem soar como duas coisas bem diferentes, mas regras são a fonte mais consistente de diversão do jogador. Podemos associar regras a ser barrado de fazer algo que realmente queremos, mas, nos jogos, nos submetemos voluntariamente às regras \cite{halfreal}.
\end{citacao}

Representação, interação, conflito e segurança são trazidos por \citeonline{artofgame} como aspectos essenciais dos jogos.  A representação diz respeito ao modo como um jogo cria um mundo fechado, autossuficiente e simplificado, com regras claras e sistemas inter-relacionados. A interação é a ideia de que o jogo permite que o jogador explore e influencie o mundo criado pelo jogo, gerando uma relação de causa e efeito. O conflito, por sua vez, aparece quando o jogador tenta alcançar um objetivo e encontra obstáculos no processo. Por fim, a segurança é o que garante que os conflitos permaneçam somente no jogo, jamais oferecendo riscos reais ao jogador.

\citeonline{artofgame} separa os jogos em cinco categorias principais: jogos de tabuleiro, jogos de cartas, jogos atléticos, jogos infantis e jogos de computador. 

\begin{itemize}
    \item \textbf{Jogos de Tabuleiro:} os jogos de tabuleiro consistem em uma superfície dividida em setores, nos quais peças movidas pelo jogador constituem as jogadas. Cada jogo pode ter diferentes regras de como o jogador pode movimentar essas peças. Um exemplo comum é o xadrez.
    \item  \textbf{Jogos de Cartas:} os jogos de cartas são jogados com um baralho de 52 cartas com diferentes naipes e símbolos (números de 1 a 10 e as letras A, J, Q e K). Os jogos giram em torno das combinações entre esses elementos. Cada jogo tem uma regra diferente para formar tais combinações.
    \item  \textbf{Jogos Atléticos:} os jogos atléticos são jogos que dependem mais da capacidade física do jogador do que da capacidade mental. Esses jogos têm regras rigorosas acerca do que o jogador pode ou deve fazer.
    \item  \textbf{Jogos Infantis:} os jogos infantis normalmente são jogados em grupos e possuem componentes mentais e físicos simples. No entanto, para os propósitos de seu livro, \citeonline{artofgame} exclui essas brincadeiras da categoria "jogo".
    \item \textbf{Jogos de Computador:} os jogos de computador são jogados em cinco tipos diferentes de máquinas, sendo elas os \textit{arcades}, \textit{hand helds} (consoles portáteis), jogos caseiros com vários programas, máquinas como o \textit{ATARI 2600}, computadores pessoais e grandes computadores \textit{mainframe}. Os computadores permitem que os jogos utilizem gráficos animados, além de agir como oponentes dentro dos \textit{games} a de depender do gênero do jogo que está sendo jogado. Além disso, os jogos de computador exigem coordenação entre mãos e olhos. 
\end{itemize}

Vale ressaltar que a definição de \citeonline{artofgame} sobre jogos de computador é válida. Entretanto, considerando a evolução da tecnologia desde então, seria interessante considerar também que os jogos de computador podem ser jogados em dispositivos eletrônicos móveis, como tablets e celulares, por exemplo.

\subsection{Narrativa em Jogos Digitais}
\label{estrutura narrativa}

\noindent É de suma importância, antes de definir o que é uma narrativa, definir o que é história e enredo. \citeonline{narrativasbarry} define história como uma sequência cronológica de eventos, enredo como uma organização causal dos eventos, e narrativa como uma história recontada em uma sequência temporal. Baseando-se nessas definições, \citeonline{narrativasbarry} ainda afirma que as narrativas são mais maleáveis do que as histórias ou os enredos.

A narrativa cinematográfica relacionada aos jogos tem sido uma grande tendência no mercado na última década. \citeonline{gamedesign} destaca que a comunidade de \textit{game design} demonstrou grande interesse em tornar os jogos mais cinematográficos por meio da narrativa. \citeonline{jenkins2004game} diz que a aplicação da teoria do cinema aos jogos pode parecer pesada e literal, muitas vezes falhando em reconhecer as profundas diferenças entre as duas mídias.

\begin{citacao}
Muitas estratégias criadas pela literatura e
apropriadas num primeiro momento pelo cinema e
pela mídia de massa, agora são amplamente utilizadas
nos jogos eletrônicos para seduzir e dirigir a
percepção do público, inserindo-o no espaço da
ficção e simulação. É importante destacar que os
jogos eletrônicos apresentam suas histórias de
maneira peculiar, isto é, um tipo específico de
narrativa, que por sua vez, pode manter relações com
outras narrativas midiáticas, combinando ficção e não
ficção \cite{alves2009crescente}.
\end{citacao}

O tempo e o espaço também são elementos importantes para o andamento de uma narrativa. \citeonline{alves2009crescente} explicam o tempo como sendo uma base para a organização da história e um auxílio para a compreensão das ações que transformam os estados, enquanto o espaço se trata da ambientação da narrativa, ou seja: onde ela se passa. De acordo com \citeonline{jenkins2004game}, ao se referir a histórias de jogos, refere-se especificamente aos jogos nos quais o jogador pode participar ou testemunhar os eventos narrativos. \citeonline{alves2009crescente} afirmam que as narrativas dos jogos têm a função de transportar o jogador para dentro daquele mundo.

\section{Inteligência Artificial}
\label{AI}
\noindent A Inteligência Artificial é considerada uma área relativamente nova, que surgiu como uma evolução dentro da computação pouco depois da Segunda Guerra Mundial, em 1956, conforme diz \citeonline{iapeter}.  Eles consideram a IA como um campo universal e dividem sua definição em quatro grupos, sendo eles: pensando como humano; agindo como seres humanos; pensando racionalmente e agindo racionalmente.

No entanto, apesar de ter começado a ser estudada academicamente apenas em 1956, no início dos anos 50 a ficção científica já tratava de máquinas inteligentes e das leis da robótica.  \citeonline{eurobo} já tratava nessa época sobre os impactos da robótica e da Inteligência Artificial na sociedade, além de destacar em vários momentos a evolução da IA dentre os contos do livro, mostrando como em dado momento ela se torna tão parecida com a inteligência humana que a moral se confunde.

\citeonline{Schank_1987} afirma que a definição de IA depende dos objetivos dos pesquisadores, uma vez que o desenvolvimento dela não está completo. Essa visão se mantém nos dias atuais, uma vez que a Inteligência Artificial está em constante evolução. Ainda segundo \citeonline{Schank_1987}, a IA busca construir uma máquina inteligente, bem como compreender a natureza da inteligência; portanto, é tanto uma ciência quanto uma tecnologia.

\citeonline{tomai} traz um contexto histórico da Inteligência Artificial. Entre os anos 50 e 70, o maior incentivo para o desenvolvimento nesta área era governamental, impulsionado pela Guerra Fria. Já nessa época, havia debate sobre como ela deveria funcionar. Havia acadêmicos que defendiam que ela deveria ser estruturada como um computador qualquer, usando estruturas como \textit{if-then-else}, enquanto outra vertente defendia que os sistemas deveriam se basear no próprio cérebro humano e nas redes neurais.

Após os anos 70, os estudos no campo de IA enfrentaram uma considerável redução, já que o governo dos Estados Unidos tornou-se cada vez mais rigoroso com o financiamento dos estudos. Nos anos 80 e 90, ainda segundo \citeonline{tomai}, os sistemas especialistas se popularizaram, mas logo mostraram ser específicos demais, o que os impedia de serem usados em outras áreas do conhecimento.

O avanço da tecnologia como um todo, bem como a melhora da infraestrutura tecnológica com o passar dos anos reforçaram o crescimento da IA, possibilitando que esse campo se desenvolvesse mais e mais, chegando em modelos cada vez mais robustos. \citeonline{tomai} divide a IA em duas categorias: fraca e forte. A fraca trata-se dos modelos nos quais apenas tarefas específicas são executadas, enquanto a forte já diz respeito às máquinas autoconscientes. Ele afirma que, até o momento, a IA encontra-se no nível fraco. 


\subsection{Inteligência Artificial Generativa}
\label{gen AI}
\noindent Com os conceitos gerais da Inteligência Artificial previamente discutidos, esta subseção dedica-se à Inteligência Artificial Generativa (\textit{Gen AI}).  Segundo \citeonline{genai}, o termo IA generativa não tem uma definição universalmente aceita, mas a comunidade científica o associa a modelos mais complexos, que geram material semelhante ao humano.

\begin{citacao}
O termo “Inteligência Artificial Generativa” está sendo utilizado em uma ampla variedade de contextos para descrever uma gama diversificada de sistemas, capacidades, aplicações e implicações. Em muitos casos, parece funcionar como um termo genérico — um rótulo aplicável a qualquer sistema de IA capaz de produzir novos conteúdos, seja texto, imagem, áudio, vídeo, código ou dados estruturados \cite{genai}.
\end{citacao}

\citeonline{ibmai} afirmam que a IA generativa é uma subcategoria da Inteligência Artificial baseada em modelos de \textit{deep learning} (aprendizado profundo), que são algoritmos que simulam tomada de decisão e aprendizado do cérebro humano. Segundo eles, a IA generativa funciona conforme mostra a figura \ref{fig:gen-ai}.

\begin{itemize}
    \item \textbf{Treinamento:} os modelos de IA generativa são treinados com um enorme volume de dados e usam técnicas de \textit{deep learning} para isso. É nessa fase que o modelo aprende os padrões presentes nos dados, permitindo que ele replique tentando simular um humano.
    \item \textbf{Ajuste:} depois de treinados, os modelos serão ajustados com dados mais específicos. Assim, o comportamento é refinado para ficar cada vez menos generalista. Além disso, existe a possibilidade de reforço humano durante o aprendizado, onde as pessoas pontuam possíveis melhoras acerca do conteúdo gerado.
    \item \textbf{Geração:} a IA generativa é capaz e gerar conteúdo original com base em comandos fornecidos pelo usuário e inclui mecanismos de \textit{feedback} para ajustes contínuos, melhorando os resultados ao longo do tempo.
\end{itemize}

\begin{figure}[h]
    \centering
    \caption{Fluxo de funcionamento da IA generativa}
    \includegraphics[width=0.5\linewidth]{imagens/fluxo_gen_ai.pdf}
    \label{fig:gen-ai}
\end{figure}
Assim, a IA generativa pode ser entendida não só como uma evolução ou uma subcategoria dos modelos tradicionais de IA, mas como um novo paradigma, capaz de criar conteúdo original que pode impactar diversas áreas do conhecimento e da sociedade. 

\section{Modelagem de Software}
\label{modelagem}
\noindent Segundo \citeonline{sommerville}, a modelagem de sistemas, ou modelagem de software, é uma representação gráfica de um sistema de \textit{software}, usada para esclarecer o que um sistema pré-existente faz e para explicar os requisitos de um novo sistema para documentá-lo e, posteriormente, implementá-lo. 

Para fazer um modelo, é necessário levantar todos os requisitos que o \textit{software} em questão exige para ser desenvolvido. \citeonline{sommerville} define os requisitos como ``as descrições do que o sistema deve fazer, os serviços que oferece e as restrições a seu funcionamento''. \citeonline{engenhariarequisitos} afirmam que a fase do levantamento de requisitos fornece as informações necessárias para nortear a modelagem de um sistema.

\citeonline{engenhariarequisitos} também destacam que a modelagem de um sistema serve não apenas para tornar a informação mais visualmente amigável, mas também para identificar informações incorretas ou redundantes, bem como lacunas.  

\citeonline{sommerville} separa os tipos de modelagem em 4: modelos de contexto, modelos de interação, modelos estruturais e modelos comportamentais.

\begin{itemize}
    \item \textbf{Modelos de Contexto:} os modelos de contexto descrevem o ambiente do sistema, incluindo sua interação com outros sistemas e com usuários (ou seja, atores externos), mas não detalham os tipos de relacionamento entre esses sistemas. Normalmente, os modelos de contexto são acompanhados por diagramas de atividades ou modelos de processos de negócio. Seu objetivo é definir quem ou o quê interage com o sistema.
    \item \textbf{Modelos de Interação:} os modelos de interação servem para representar as dinâmicas de comunicação entre cada componente do sistema, usuários e outros sistemas. \citeonline{sommerville} cita os diagramas de caso de uso, que modelam as interações do sistema com atores externos, e diagramas de sequência, que servem para representar as interações internas. Seu objetivo é definir como as interações com o sistema acontecem.
    \item \textbf{Modelos Estruturais:} os modelos estruturais mostram como o sistema se organiza, mostrando seus componentes (como classes, objetos, entre outros) e os relacionamentos entre eles. Eles servem para compreender como a arquitetura do \textit{software} é construída antes que o sistema seja, de fato, implementado.
    \item \textbf{Modelos Comportamentais:} os modelos comportamentais modelam o comportamento de um sistema em tempo de execução. Isso inclui fluxo de processos, respostas a eventos e interações dinâmicas entre componentes. Esse tipo de modelagem antecipa possíveis erros antes da implementação, documentam a lógica para manutenção posterior e permitem a simulação de cenários diversos e complexos. Seu objetivo é mostrar como o sistema reage a entradas, eventos ou mudanças de estado. As redes de Petri, que serão detalhadas logo a seguir, se encaixam nessa categoria.
\end{itemize}
\subsection{Redes de Petri}
\label{petri}
\noindent Como explicado na seção anterior, as redes de Petri são um tipo de modelagem de \textit{software}. \citeonline{petri-net} as define como uma ferramenta matemática de estudo e representação de sistemas, que pode revelar aspectos importantes sobre a estrutura e o comportamento do sistema modelado. Elas foram projetadas para modelar sistemas com componentes concorrentes que interagem entre si.

As redes de Petri surgiram a partir da tese de Carl Adam Petri\footnote[1]{Carl Adam Petri foi um matemático e cientista da computação alemão. Ele inventou as redes de Petri aos treze anos, em 1939, e as documentou vinte e três anos depois, em 1962.}, \textit{``Kommunikation mit Automaten''} (Comunicação com Autômatos), de 1962. \citeonline{petripaulo} elucidam que, ao chamar a atenção de outros pesquisadores, Petri e esses demais desenvolveram boa parte da teoria, notação e representação das redes de Petri. 

\citeonline{petri-net} define as redes de Petri como uma 4-upla $C = (P, T, I, O)$, onde:

\begin{itemize}
    \item \textbf{P =} ${p_1, p_2, ..., p_n}$: um conjunto finito de lugares com $ n \geq 0$;
    \item  \textbf{T = } ${t_1, t_2, ..., t_m}$: um conjunto finito de transições com $ m \geq 0$;
    \item \textbf{I = } função de entrada \textit{(input function)} $\rightarrow$ mapeia transições para conjuntos (ou \textit{bags}) de lugares;
    \item \textbf{O = } função de saída \textit{(output function)} $\rightarrow$ mapeia transições para conjuntos (ou \textit{bags}) de lugares.
\end{itemize}

A figura \ref{fig:4-upla} mostra uma rede de Petri descrita pela quádrupla $C = (P, T, I, O)$.


\begin{figure}[h]
    \centering
    \caption{Rede de Petri representada pela quádrupla}
    \includegraphics[width=0.5\linewidth]{imagens/fig_2_1_peterson.png}
    \caption*{Fonte: \cite{petri-net}}
    \label{fig:4-upla}
\end{figure}


Essa mesma estrutura descrita pela quádrupla ilustrada pela figura \ref{fig:4-upla} pode ser representada por um multigrafo bipartido e dirigido. Um multigrafo é um tipo de grafo onde podem haver várias arestas entre o mesmo par de nós. Por ser dirigido, suas arestas têm uma direção definida, e por ser bipartido, seus nós são divididos em dois conjuntos disjuntos (no caso das redes de Petri, lugar e transição), e as arestas só se ligam a nós de conjuntos distintos.

Um círculo (O) representa um lugar; uma barra (|) representa uma transição. Como os círculos representam graficamente os lugares, eles serão denominados lugares. Da mesma forma, as barras serão denominadas transições. De acordo com \citeonline{petri-net}, um lugar $p_i$ é considerado um lugar de entrada da transição $t_j$ se $p_i$ está presente no conjunto de entrada da transição $t_j$. De modo análogo, a presença de $p_i$ no conjunto de saída da transição $t_j$ configura $p_i$ como um lugar de saída \cite{petri-net}.

Os \textit{tokens} são elementos que, associados aos lugares, representam o estado atual do sistema modelado. A distribuição deles nos lugares é chamada de marcação. Formalmente, uma marcação \textit{M} é uma função que associa um número não negativo de \textit{tokens} $M(p)$ a cada lugar $p \in P$. O número de \textit{tokens} consumidos ou produzidos por uma transição é determinado pela multiplicidade do lugar nas funções de entrada e saída. A figura \ref{fig:grafo-petri} ilustra a rede descrita pela quádrupla da figura \ref{fig:4-upla}. Ela contém lugares, transições e \textit{tokens}.

\begin{figure}[h]
    \centering
    \caption{Representação gráfica da rede de Petri da \autoref{fig:4-upla}}
    \includegraphics[width=0.5\linewidth]{imagens/rede_petri_marcada_2_1_peterson.png}
    \caption*{Fonte: \cite{petri-net}.}
    \label{fig:grafo-petri}
\end{figure}


Na notação de \citeonline{petri-net}, as funções $I$ e $O$ mapeiam transições para os conjuntos de lugares, permitindo que um lugar apareça mais de uma vez. A multiplicidade de um lugar $p$ em $I(t)$, denotada por $\#(p, I(t))$, indica quantos \textit{tokens} a transição $t$ consome de $p$. De forma análoga, a multiplicidade $\#(p, O(t))$ indica quantos \textit{tokens} $t$ produz em $p$. Uma transição $t$ está habilitada se, para todo $p \in I(t)$, $M(p) \geq \#(p, I(t))$.  O conceito de multiplicidade muito se assemelha ao conceito de peso que \citeonline{janettepetri} trazem, com a diferença que \citeonline{petri-net} não representa os ``pesos'' graficamente, como rótulos dos grafos. 

Além de sua estrutura formal, as redes de Petri permitem a análise de propriedades fundamentais dos sistemas modelados. Tais propriedades podem ser classificadas como estruturais — independentes da marcação inicial — e comportamentais — que dependem diretamente da evolução do sistema ao longo do tempo. Segundo \citeonline{murata-rede-petri}, essas propriedades incluem, mas não estão limitadas a alcançabilidade, vivacidade, reversibilidade e limitabilidade.

\begin{itemize}
    \item \textbf{Alcançabilidade}: a alcançabilidade é uma propriedade que analisa se uma marcação específica $M_n$ pode ser alcançada a partir de uma marcação inicial $M_0$ por meio de uma sequência de disparos de transições $t$. $M_n$ é alcançável se existe uma sequência de disparos $\sigma = t_1, t_2, t_3 ... t_n$ tal que $M_0$ é transformada em $M_n$.
    \item \textbf{Vivacidade}: a vivacidade está relacionada à ausência de \textit{deadlocks}. Uma transição é considerada viva se, em qualquer marcação alcançável a partir da marcação inicial $M_0$, ela sempre pode ser disparada novamente após alguma sequência de disparos, não importa qual seja a sequência disparada. Formalmente, uma Rede de Petri $(N, M_0)$ é viva se todas as suas transições são vivas. Para sistemas complexos, a vivacidade pode ser relaxada em níveis hierárquicos
\[ \text{L4-viva} \Rightarrow \text{L3-viva} \Rightarrow \text{L2-viva} \Rightarrow \text{L1-viva}, \]
onde:
\begin{itemize}
    \item \textbf{L0-viva (morta):} uma transição $t$ é L0-viva se nunca pode ser disparada em nenhuma sequência de disparos $L (M_0)$.
    \item \textbf{L1-viva (potencialmente disparável): }uma transição $t$ é L1-viva se pode ser disparada pelo menos uma vez em alguma sequência $L (M_0)$.
    \item \textbf{L2-viva:} uma transição $t$ é L2-viva se, para qualquer inteiro positivo $k$, existe uma sequência onde $t$ é disparada pelo menos $k$ vezes.
    \item \textbf{L3-viva:} uma transição $t$ é L3-viva se existe uma sequência infinita onde $t$ aparece infinitamente.
    \item \textbf{L4-viva (ou apenas viva):} uma transição $t$ é L4-viva se é L1-viva para todas as marcações $M \in R (M_0)$.

 
\end{itemize}
    \item \textbf{Reversibilidade}: uma Rede de Petri é reversível se, a partir de qualquer marcação alcançável $M \in R (M_0)$, é possível retornar à marcação inicial $M_0$. Isso significa que o sistema pode sempre ser reinicializado. 
\begin{itemize}
        \item \textbf{Estado \textit{Home:}} é uma marcação $M'$ que pode ser alcançada a partir de qualquer outra marcação em $M \in R (M_0)$. Diferente da reversibilidade, não é necessário retornar ao estado inicial, apenas a um estado específico $M'$.
    \end{itemize}
    \item \textbf{Limitabilidade}: uma Rede de Petri é dita $k$-limitada se o número de \textit{tokens} em cada lugar não excede um valor finito $k$ para qualquer marcação alcançável a partir de $M_0$. Se $k=1$ a rede é chamada de segura (\textit{safe}), ou seja: garante que nenhum lugar terá mais de 1 \textit{token}, evitando conflitos por recursos compartilhados. Essa propriedade é essencial para garantir que \textit{buffers} ou registros em sistemas modelados não sofram \textit{overflow} \footnote[2]{Se um valor excede a capacidade de armazenamento ou processamento de um sistema, é dito que aconteceu um \textit{overflow}.}, independentemente da sequência de disparos executada. 
\end{itemize}

Vale ressaltar que as propriedades são independentes entre si. Por exemplo, uma rede pode ser reversível, mas não viva ou limitada, e vice-versa.



\chapter{Trabalhos relacionados}\label{cap:relacionados}

\noindent Neste capítulo serão abordados os trabalhos que detêm assunto de pesquisa semelhante ao proposto por este trabalho. A partir disto, foram realizadas diversas buscas nas bases de dados de pesquisas \textit{on-line}, como, por exemplo, o \textit{IEEE Xplore}, o \textit{ACM Digital Library}, bem como o  \textit{SciELO}. Essas buscas tiveram como objetivo buscar artigos científicos, dissertações de mestrado e teses de doutorado que abordassem o tema de geração de narrativas com o uso de Inteligência Artificial com as seguintes \textit{strings} de busca: \textit{``AI generated storytelling''}, \textit{``generating narratives with AI''}, \textit{``game narrative AI''} e \textit{``petri net game story''}. Um dos principais critérios de seleção foi a proximidade com o tema tratado neste trabalho, uma vez que, caso muito ampla, a pesquisa retornaria artigos que o deixariam tangenciado. Foi dada preferência para trabalhos que estivessem escritos nos idiomas Inglês e Português. Por fim, a seguir, estão os trabalhos encontrados que mais se relacionam com essa pesquisa.

\section{\textit{Unveiling New Realms: Enhancing Procedural Narrative Generation and NPC Personalization using AI}}
\noindent Com o avanço da Inteligência Artificial (IA), a geração de narrativas procedurais em jogos tem ganhado destaque, mas ainda enfrenta desafios como falta de profundidade emocional, repetitividade e incoerência narrativa \cite{chawla24}. \citeonline{chawla24} propõe um framework de design de sistema que utiliza modelos GPT \textit{(Generative Pre-trained Transformers)} que visa aprimorar a geração de narrativas e a personalização de personagens não jogáveis \textit{(NPCs)} em jogos, visando criar experiências de jogo mais ricas e personalizadas para o usuário.

O estudo busca abordar as limitações da geração procedural de narrativas. utilizando IA generativa para criar histórias dinâmicas e coerentes. Para isso, o autor propõe um sistema que combina estruturas narrativas bem definidas com diretrizes que garantem que o conteúdo gerado seja coeso. O sistema é implementado em um jogo de cartas de mistério criminal, onde a narrativa, os diálogos dos NPCs e os elementos do enredo são gerados de modo dinâmico pelo GPT. 

A metodologia usada é a criação de um \textit{framework} de design de sistema que se apropria de técnicas como \textit{chunking} (divisão de informações em partes menores) e geração de dados em formato JSON para garantir que o conteúdo gerado seja relevante e pronto para uso no jogo. O sistema é projetado para manter a consistência das personalidades dos NPCs e evitar a geração de conteúdo inadequado ou fora do contexto. O jogo desenvolvido, chamado "\textit{Dark Shadows}", utiliza \textit{loops} de jogabilidade que permite ao jogador resolver casos, coletar itens e desvendar a narrativa principal. Todos os elementos narrativos são 
\newpage
\vspace*{0.01cm}gerados em tempo real pelo GPT.

\citeonline{chawla24} conclui que o uso de modelos GPT juntamente com um \textit{framework} bem estruturado é capaz de trazer significativa melhora para a geração de narrativas procedurais e para a personalização de NPCs em jogos. É enfatizada a importância de certo pragmatismo ao se deparar com as limitações dos modelos de linguagem, sendo necessário o constante ajuste nos \textit{prompts}, garantindo a conformidade dos eventos narrados. Por fim, \citeonline{chawla24} abre caminho para futuras pesquisas em narrativas interativas e design de jogos, sugerindo que modelos GPT personalizados podem ser uma ferramenta poderosa para criar experiências de jogo mais imersivas e dinâmicas.

\section{\textit{Plans and planning in narrative generation: a review of plan-based approaches to the generation of story, discourse and interactivity in narratives}}
\noindent Pesquisas computacionais voltadas para a construção de modelos de narrativa e seu uso têm sido interesse de vários pesquisadores há bastante tempo. Esse interesse tem sido influenciado por diversas disciplinas, entre elas a teoria da narrativa, os estudos de jogos e até mesmo a psicologia cognitiva, bem como estudos de cinema \cite{young2013plans}. O trabalho de \citeonline{young2013plans} foca em três aspectos e os coloca como os principais: história, discurso e interatividade. 

A história inclui eventos, ações e personagens. O discurso se refere ao modo como conta-se a história, desde escolhas de vocabulário até escolhas de técnicas cinematográficas \cite{young2013plans}. Por fim, a interatividade diz respeito às narrativas geradas em ambientes interativos, onde a história pode ser influenciada. Um exemplo de cenário interativo trazido pelos autores são os jogos, onde a geração de histórias envolve a criação de sequências de ações que formam o enredo. 

O planejamento de narrativa baseado em IA é bastante útil para modelar a coerência narrativa, equilibrar os objetivos dos personagens com os do autor e representar conflitos. \citeonline{young2013plans} trazem exemplos de sistemas pioneiros na geração de histórias baseadas em objetivos de personagens, assim como aqueles que buscavam equilibrar os objetivos dos personagens com as intenções do autor. 

O artigo também destaca a possibilidade de utilizar algoritmos de planejamento para gerar suspense e surpresa. Sistemas que usam esses algoritmos simulam o processo de compreensão do leitor, usando modelos de planejamento para prever como o público vai interpretar a narrativa. As mesmas técnicas de planejamento podem ser usadas para gerar recursos visuais, como controle de câmeras que criam efeitos específicos (como o próprio suspense).

Os autores afirmaram que a geração de narrativas interativas, como em jogos, apresenta desafios únicos, especialmente no que diz respeito ao paradoxo narrativo que envolve a tensão entre a estruturação da história e a liberdade de ação do jogador. No trabalho de \citeonline{young2013plans}, o planejamento pode ser usado para controlar personagens autônomos ou gerar enredos que se adaptam às ações do jogador. 

\citeonline{young2013plans} concluem que tais abordagens têm sido amplamente usadas em ambientes interativos e não interativos e têm facilitado a geração de histórias, discursos e interatividade. Entretanto, ainda há desafios a serem superados, como a representação de conceitos narrativos mais complexos, como narradores não confiáveis e focalização. Ainda assim, os autores acreditam que as abordagens baseadas em planejamento continuarão a ser fundamentais para a criação de conteúdo narrativo rico e envolvente.

\section{\textit{Language as Reality: A Co-creative Storytelling Game Experience in 1001 Nights Using Generative AI}}
\noindent \citeonline{sun2023} apresentam \textit{``1001 Nights''}, um jogo narrativo que utiliza Inteligência Artificial Generativa \textit{(GenAI)}, incluindo modelos de geração de imagens e linguagem (LLMs). Inspirado na ideia de que "os limites da minha linguagem significam os limites do meu mundo", de Wittgenstein\footnote[3]{Ludwig Joseph Johann Wittgenstein foi um dos filósofos mais influentes do século XX. O trabalho de \citeonline{sun2023} cita a frase \textit{"The limits of my language are the limits of my world", da obra \textit{"Tractatus Logico-Philosophicus"} (1921), que discute a relação entre linguagem, pensamento e realidade.}}, o jogo explora o conceito de linguagem como realidade. A protagonista tem o poder de transformar palavras em objetos reais dentro do jogo, podendo usá-los como armas contra o Rei. 

O jogo é dividido em duas fases: narrativa e batalha. Na fase de narrativa, os jogadores guiam o Rei a contar histórias que contêm palavras-chave, que se materializam como armas no jogo. Conforme as armas são coletadas, o mundo da história começa a invadir a realidade do jogo, criando uma fusão entre narrativa e \textit{gameplay}. O objetivo final é reescrever o destino da protagonista. \citeonline{sun2023} exemplificam o conceito de jogos nativos de IA, onde a \textit{GenAI} não é apenas um recurso adicional, mas é fundamental para a mecânica e a existência do jogo. 

O artigo propõe o termo AI-Nativo para categorizar jogos onde a GenAI é essencial para a mecânica e existência do jogo. Diferente dos jogos baseados em IA tradicional, que utilizam técnicas como algoritmos de \textit{path-finding} ou árvores de decisão, os jogos AI-Nativos dependem da geração de conteúdo em tempo real, como diálogos e imagens, que não são predefinidos pelos desenvolvedores.

Um dos principais desafios foi garantir que os jogadores se engajassem ativamente na criação de histórias, evitando entradas aleatórias ou sem sentido. Para isso, foi implementado um sistema de avaliação de histórias usando o GPT-4, onde o Rei avalia a validade da história e fornece feedback em tempo real. Se a história for considerada inválida, o Rei solicita que o jogador a reescreva, mantendo a coerência narrativa.

\citeonline{sun2023} dizem que, embora existam desafios, como a imprevisibilidade dos modelos de IA, o jogo demonstra como a IA generativa pode ser usada para criar mecânicas de jogo inovadoras e experiências narrativas diferentes das que já existem. O artigo sugere que, à medida que a IA generativa evolui, novos métodos e abordagens serão necessários para equilibrar a liberdade criativa dos jogadores com a consistência narrativa e a imersão no jogo.

\section{\textit{Petri-nets for game plot}}

\noindent O trabalho de \citeonline{brom2006petri} apresenta uma forma inovadora de criar um enredo não linear e de gerenciar a história de acordo com esse enredo utilizando as redes de Petri. Os autores propõem uma técnica com a qual a narrativa pode evoluir de diversas formas com as ações do jogador, sem perder a coerência.

A técnica utiliza dois modelos de redes de Petri: um protótipo e um modelo final. O protótipo serve para validar a trama em uma aplicação teste, sem que haja a necessidade de que o ambiente virtual esteja completo. O modelo final, por sua vez, integra eventos do ambiente virtual, como as ações do jogador ou o comportamento dos \textit{NPC's} (personagens não jogáveis), com o intuito de que a narrativa se adapte em tempo real.

\citeonline{brom2006petri} enfatizam que a solução criada é especialmente eficaz para jogos com ambientes virtuais robustos e de longa duração, onde mais de um evento pode ocorrer em múltiplos espaços e simultaneamente. Ao combinar o controle de alto nível da narrativa, por meio das redes de Petri, com a autonomia dos \textit{NPCs}, que são guiados por planos reativos hierárquicos, a história é capaz de avançar de forma estruturada e mantendo a liberdade dos personagens virtuais.

Um dos desafios explorados por \citeonline{brom2006petri} é a complexidade visual que as redes de Petri assumem ao serem usadas para tramas mais elaboradas. A fim de contornar esse problema, os autores sugerem a utilização do \textit{Microsoft Visio} para simplificar a representação dos enredos. Além disso, eles destacam a importância da realização de testes iterativos durante o design da narrativa.

Por fim, os autores sugerem que essa abordagem pode ser integrada a \textit{frameworks} como o \textit{IVE (Intelligent Virtual Environment)}, podendo assim expandir suas aplicações em jogos ainda mais interativos, além de abrir caminho para pesquisas futuras que combinem métodos formais de design narrativo com técnicas de IA para criar experiências mais imersivas.

\section{Considerações Finais}
\label{comparacao}

\noindent Este trabalho destaca-se por três contribuições principais em relação aos estudos existentes. Primeiramente, é o único a propor o uso de redes de Petri especificamente para validar narrativas geradas por IA, garantindo coerência em histórias produzidas automaticamente. Enquanto \citeonline{brom2006petri} empregam redes de Petri para modelar enredos pré-definidos, este trabalho usa a Rede de Petri como uma ferramenta de validação formal para conteúdo gerado por Inteligência Artificial a partir de um roteiro elaborado durante a pesquisa.

Em segundo lugar, este trabalho estabelecerá uma abordagem metodológica clara e replicável, com um roteiro passo a passo que vai desde a geração da narrativa até sua validação estrutural. Isso contrasta com soluções pontuais, como o uso direto de GPT em \citeonline{chawla24}, ou discussões teóricas sobre planejamento narrativo, como em \citeonline{young2013plans}.

Por fim, diferencia-se por priorizar ferramentas de IA generativa gratuitas e acessíveis, tornando o método viável para desenvolvedores independentes. Enquanto trabalhos como \citeonline{sun2023} e \citeonline{chawla24} dependem de modelos proprietários (ex.: GPT-4), este trabalho demonstra como alcançar resultados semelhantes com recursos abertos, ampliando o acesso à tecnologia.

\begin{figure}[H]
    \centering
    \caption{Tabela de Comparação de Trabalhos Relacionados}
    \includegraphics[width=0.95\linewidth]{imagens/tabalhosrel.png}
    \caption*{}
    \label{trabalhosrelacionados}
\end{figure}

A figura \ref{trabalhosrelacionados} ilustra a comparação dos trabalhos relacionados com a presente pesquisa. Note que o trabalho de \citeonline{young2013plans} não é marcado na tabela por se tratar de um trabalho com maior enfoque em revisão teórica (que foi importante para a fundamentação desta pesquisa), enquanto \citeonline{sun2023}, \citeonline{brom2006petri} e \citeonline{chawla24} trazem implementações concretas.


\chapter{Procedimentos de Pesquisa}\label{cap:implementacao}
\noindent Este capítulo descreve o processo de concepção, construção e validação das narrativas geradas. Primeiramente, serão definidas as características da narrativa e a forma como os \textit{prompts} serão elaborados com base nesses atributos. Em seguida, serão expostos os exemplos desenvolvidos, detalhando os \textit{prompts} utilizados e as ferramentas empregadas em sua criação. Por fim, serão apresentados os resultados obtidos, incluindo a análise do material gerado, os critérios de aproveitamento e o processo de validação realizado.

\section{Critérios de Narrativa}
\label{criterios-narrativa}
\noindent Conforme mencionado anteriormente, as narrativas de jogo têm apresentado a tendência de se tornarem cada vez mais cinematográficas. Por esse motivo, também foram investigadas referências nos estudos de cinema, a fim de definir as características essenciais de uma narrativa.

\citeonline{progressao} definem três recursos fundamentais para a escrita, sendo eles: espaço, tempo e ação. A partir da definição desses três elementos, o autor escolhe um modelo de roteiro, e então são acrescentados outros elementos importantes, como gênero, criação e caracterização de personagens. Segundo \citeonline{manualroteiro}, "o roteiro é uma história contada em imagens, diálogos e descrições, localizada no contexto da estrutura dramática". O paradigma básico de um roteiro, ou seja, o esquema conceitual do roteiro utilizado neste trabalho é o paradigma descrito na figura \ref{paradigma}.

\begin{figure}[H]
    \centering
    \caption{Paradigma de um roteiro}
    \includegraphics[width=0.70\linewidth]{imagens/paradigma_roteiro.png}
    \caption*{Fonte: \cite{manualroteiro}}
    \label{paradigma}
\end{figure}

Ainda de acordo com \citeonline{manualroteiro}, os três atos se decompõem, de maneira sucinta, da seguinte forma:

\begin{itemize}
    \item \textbf{Ato I (Apresentação)}: momento em que se apresentam o protagonista, o mundo em que vive, os personagens secundários e a premissa, o tema central da história. O espectador (ou, no caso deste trabalho, o jogador) deve entender rapidamente “quem é quem” e o que está acontecendo no momento em que a narrativa se encontra. O Ato I termina com o primeiro ponto de virada, um acontecimento que muda o rumo da narrativa e conduz à ação principal.
    \item \textbf{Ato II (Confrontação)}: é o núcleo da história, onde o personagem principal enfrenta uma série de obstáculos e conflitos que o afastam de seu objetivo, revelando sua força e fraquezas. Aqui ocorre o desenvolvimento das relações, tensões e dilemas centrais. O ato se encerra com o segundo ponto de virada, um evento que redireciona a trama para a resolução.
    \item \textbf{Ato III (Resolução)}: diz respeito à solução do conflito e o desfecho da história. O protagonista atinge (ou não) sua meta, e as consequências de suas escolhas são reveladas. “Resolução” não significa apenas o fim, mas a resposta dramática ao que foi proposto no início. É o momento de amarrar as linhas narrativas e deixar uma impressão final no público.
\end{itemize}

\citeonline{Lindley2005StoryAN} também define os mesmos três atos em um trabalho sobre jogos, o que evidencia como os campos se entrelaçam. Assim, desenha-se o primeiro e mais essencial critério de narrativa: ela deve conter início, meio e fim, seguindo o fluxo "modelo-confronto-resolução", descrito por \citeonline{manualroteiro}.

Na construção de uma narrativa para um jogo eletrônico, deve-se levar em conta as seguintes questões: qual o objetivo do jogo? O que o jogador
terá de fazer? Como ele vai fazer? O que tornará o jogo divertido? Quem será o inimigo? Como criar um personagem principal de forma que atraia o público? \cite{beatriz2019narrativa}. Essas questões são importantes de serem feitas, já que, diferente de uma narrativa cinematográfica, uma narrativa de jogo precisa dar abertura para interatividade. Por isso, neste trabalho, serão utilizados conceitos tanto dos jogos quanto do cinema.

Os elementos essenciais de uma narrativa, segundo \citeonline{Lindley2005StoryAN}, são:

\begin{itemize}
    \item \textbf{História, enredo e discurso} (três níveis de estrutura narrativa);
    \item \textbf{Protagonista, conflito e resolução} (núcleo da ação dramática);
    \item \textbf{Três atos principais} (apresentação, confronto e desfecho);
    \item \textbf{Temporalidade e causalidade}, que conectam os eventos e dão coerência ao enredo.
\end{itemize}
\section{Construção do \textit{Prompt}}
\label{prompt}
\noindent A partir dos elementos definidos na seção anterior, foi possível criar os \textit{prompts} (conjuntos de instruções) que serviram de entrada para os modelos de IA que foram testados. Os \textit{prompts} foram criados combinando os requisitos de uma narrativa com engenharia de \textit{prompt}. A engenharia de \textit{prompt} é o processo de projetar, refinar e otimizar \textit{prompts} de entrada para comunicar efetivamente a intenção do usuário a uma Inteligência Artificial \cite{Ekin_2023}.

Segundo \citeonline{gao2023prompt}, o \textit{roleprompting} é uma maneira simples de incentivar a geração de textos em um estilo criativo específico, como o de um autor. Essa técnica, chamada em português de personificação, consiste em atribuir um papel ao modelo antes de solicitar uma tarefa. Ela é capaz de aumentar o foco e a precisão da saída da IA, além de reduzir ambiguidades no comando. Sendo assim, essa foi a técnica escolhida para a construção do \textit{prompt}. 

Para ilustrar a aplicação prática da técnica de personificação, bem como aplicar os conceitos referentes a narrativas, apresenta-se o \textit{prompt} inicial desenvolvido para este trabalho:

\textit{''A estrutura narrativa de um filme, normalmente, segue o seguinte padrão:}


\textit{\textbf{1 - Exposição}: Apresentação do mundo, personagens e situação inicial;}

\textit{\textbf{2 - Incidente incitant}e: Evento que desencadeia a história principal;
}

\textit{\textbf{3 - Desenvolvimento/Confrontação}: Escalada do conflito e obstáculos;}

\textit{\textbf{4 - Clímax}: Momento de maior tensão e decisão crucial;}

\textit{\textbf{5 - Resolução/Desfecho}: Conclusão dos conflitos e nova situação;}


\textit{A trama envolve uma gama de personagens, sendo eles o protagonista, o antagonista e os personagens secundários. É comum que haja algum tipo de arco ou curva de personagem mostrando uma grande mudança no protagonista ou antagonista. Pode ser um arco de redenção, a jornada do herói, etc.
A narrativa, por sua vez, é movida por um conflito, que pode envolver vários personagens ou ser um conflito interno/psicológico do personagem principal.
É comum que haja algum tipo de tema central ou subtexto abordado, dando mais profundidade a história e aos personagens abordados.
Outro elemento extremamente importante é a ambientação: tempo e espaço, onde e quando a história se passa. }

\textit{Sabe-se que atualmente as narrativas de jogos têm se tornado cada vez mais cinematográficas. Isso permite que, ao delimitar os elementos essenciais para uma narrativa de jogo, conceitos do cinema possam ser amplamente utilizados.
}
\textit{Tendo tudo isso em mente: }

\textit{Você é um game designer que cria roteiros/narrativas para jogos. Sua missão é criar uma narrativa para um jogo de <gênero>, sobre <assunto>, com um protagonista <fem/masc> de origem <etnia>, chamado <nome>. Lembre: o protagonista deve ter um perfil definido (nome, origem, histórico, motivação). Atenha-se às etapas 1 e 2 do padrão que o cinema utiliza. Lembre-se de fazer um planejamento, apresentando o personagem e sua história de background, como uma espécie de perfil, e depois escreva a história de uma fase inicial e introdutória do jogo, que, em um filme, seria como a cena de abertura + parte do primeiro ato. O tom deve ser <atmosfera>. }

\textit{História base:}

\textit{<história>''}


O \textit{prompt} apresentado configura-se como um modelo base, ou \textit{template}, destinado à geração de narrativas em ambiente de Inteligência Artificial generativa. Sua estrutura foi concebida de modo a padronizar as instruções dadas ao modelo, garantindo consistência nas respostas e possibilitando a análise comparativa entre as saídas geradas. Esse \textit{prompt} foi empregado na etapa de testes, na qual foram avaliados aspectos como coerência narrativa, fidelidade aos critérios definidos na seção \ref{criterios-narrativa} e o potencial criativo das narrativas resultantes.

A composição do modelo segue uma estrutura simples, mas eficaz. Em um primeiro momento, apresenta-se um conjunto de conceitos fundamentais sobre narrativa, extraídos de referenciais teóricos do cinema e dos jogos, que funcionam como uma espécie de treinamento para a Inteligência Artificial. Essa introdução fornece ao modelo um repertório básico sobre a forma narrativa, permitindo que ele reconheça elementos estruturais essenciais, como exposição, conflito, clímax e desfecho.

Na sequência, o \textit{prompt} orienta o modelo por meio de um roteiro que organiza o processo criativo, especificando parâmetros como gênero, tema, perfil do protagonista, atmosfera e recorte narrativo desejado. Somente após essa contextualização teórica e estrutural são introduzidos os elementos variáveis, isto é, as informações específicas de cada teste narrativo, o que garante que a geração textual mantenha coerência e aderência aos princípios previamente definidos.

Entretanto, é importante destacar que a formulação do \textit{prompt} pressupõe que a pessoa responsável pela inserção dos dados possua um conhecimento mínimo sobre os conceitos narrativos, assim como sobre o funcionamento da IA. Essa familiaridade teórica é o que permite compreender a função de cada etapa do \textit{prompt} e, se necessário, adaptar suas definições de acordo com o objetivo narrativo desejado. Usuários sem esse repertório tendem a reproduzir o modelo de forma mecânica, sem explorar plenamente seu potencial interpretativo e criativo.

Dessa forma, essa condição (ter conhecimento prévio) é importante para que o \textit{prompt} funcione não apenas como uma ferramenta técnica de geração textual, mas também como um instrumento metodológico de experimentação e análise das capacidades narrativas da IA.

 

\section{Geração da Narrativa}
\label{geracao}

\noindent Com o \textit{prompt} definido na seção anterior, procedeu-se à geração das narrativas utilizando alguns LLMs (\textit{Large Language Model} - modelo de linguagem de grande escala), sendo eles \textit{ChatGPT} na versão GPT-5, \textit{Deepseek V3} e \textit{Gemini 2.5 Flash}. Essa etapa teve como objetivo avaliar a capacidade dos modelos de IA em compreender as instruções fornecidas e produzir textos narrativos coerentes, estruturados e alinhados aos critérios estabelecidos na figura ~\ref{criterios}.

\begin{figure}[H]
    \centering
    \caption{Parte 1 da Tabela de Comparação dos LLMs}
    \includegraphics[width=0.95\linewidth]{imagens/criterios1.jpg}
    \caption*{}
    \label{criterios1}
\end{figure}

\begin{figure}[H]
    \centering
    \caption{Parte 1 da Tabela de Comparação dos LLMs}
    \includegraphics[width=0.95\linewidth]{imagens/criterios2.jpg}
    \caption*{}
    \label{criterios2}
\end{figure}

Os elementos apresentados nas figuras \ref{criterios1} e \ref{criterios2} representam, por meio de uma tabela, aspectos fundamentais da construção narrativa clássica, de acordo com a literatura, e foram adotados como critérios para análise da qualidade das saídas produzidas pelos modelos de IA. Esses elementos funcionam como um \textit{checklist}, de modo que uma saída ideal deve contemplar todos eles. 

Cada um dos elementos tem sua importância. A \textbf{exposição}, por exemplo, estabelece a base emocional e situacional da trama, permitindo que o leitor compreenda as motivações e o cenário antes do conflito principal, ao passo que o \textbf{incidente incitante} funciona como gatilho dramático, despertando o interesse e definindo o objetivo do protagonista. O \textbf{desenvolvimento} trata-se do eixo dramático que mantém a tensão e promove a progressão da história, enquanto o \textbf{clímax} representa o ponto de maior tensão e conflito, onde ocorre a decisão mais importante ou o enfrentamento final do protagonista e do antagonista (seja este físico ou metafísico). Por fim, a \textbf{resolução} é o encerramento da narrativa, onde os conflitos são solucionados e o equilíbrio é restaurado.

Dessa forma, os testes foram realizados ao longo de três semanas, entre agosto e setembro de 2025, utilizando os LLMs mencionados anteriormente: \textit{ChatGPT} (versão GPT-5), \textit{Deepseek V3} e \textit{Gemini 2.5 Flash}. O objetivo principal foi avaliar a capacidade de cada modelo em compreender o \textit{prompt} desenvolvido e gerar narrativas coerentes, priorizando os dois primeiros momentos da estrutura narrativa (a exposição e o incidente incitante) conforme definido nos critérios de análise.

Para garantir a imparcialidade dos resultados, foi criada uma conta de usuário nova para cada modelo testado, de modo a evitar qualquer interferência de histórico de conversas, preferências prévias ou dados contextuais que pudessem influenciar a produção textual. Em todos os casos, o mesmo \textit{prompt} foi inserido integralmente, sem modificações de conteúdo, para assegurar as mesmas condições em cada IA.

Em cada sessão, a interação com a IA foi limitada a uma única solicitação de geração narrativa, sem refinamentos ou correções adicionais no mesmo diálogo. Dessa forma, foi possível avaliar o desempenho de cada modelo a partir da primeira resposta fornecida, o que contribui para a objetividade da comparação entre as saídas. Eventuais ajustes no \textit{prompt} foram realizados apenas após o término de cada rodada de testes, sempre visando aprimorar a clareza das instruções, e não o conteúdo narrativo em si.

As narrativas produzidas por cada modelo foram analisadas com base nos critérios definidos anteriormente. As tabelas apresentadas nas figuras \ref{criterios-preenchida1} e \ref{criterios-preenchida2} exemplifica a comparação de uma das semanas e permite visualizar o desempenho relativo de cada IA nos diferentes aspectos avaliados.

\begin{figure}[H]
    \centering
    \caption{Critérios atendidos das narrativas geradas pelas IAs}
    \includegraphics[width=0.95\linewidth]{imagens/criteriospreenchida1.jpg}
    \caption*{}
    \label{criterios-preenchida1}
\end{figure}

\begin{figure}[H]
    \centering
    \caption{Critérios atendidos das narrativas geradas pelas IAs}
    \includegraphics[width=0.95\linewidth]{imagens/criteriospreenchida2.jpg}
    \caption*{}
    \label{criterios-preenchida2}
\end{figure}

Dentre os modelos testados, o \textit{Deepseek V3} apresentou o melhor desempenho geral. Suas saídas demonstraram uma ambientação mais rica, excelente contextualização da situação inicial e um domínio expressivo da atmosfera do gênero escolhido, oferecendo uma experiência narrativa mais coesa e envolvente. Ainda que a escolha da melhor narrativa envolva certa dimensão interpretativa, o processo de análise foi conduzido de maneira sistemática e comparativa, reduzindo a influência de preferências pessoais e assegurando maior imparcialidade nos resultados.

Vale ressaltar que é altamente recomendado ter uma ideia pré-existente para replicar este trabalho, pois, levando em conta que a IA opera recombinando e ressignificando ideias existentes em seu treinamento, existe a possibilidade de acontecer um plágio, ainda que não intencional, do trabalho de outrem. Para mitigar esse risco e assegurar a originalidade do trabalho, é fundamental que o usuário forneça um \textit{prompt} detalhado e estruturado a partir de uma ideia narrativa própria e bem definida.

\section{Validação com Rede de Petri}
\label{validacao}

\noindent Com a narrativa gerada em mãos, foi possível criar um modelo de rede de Petri capaz de representar graficamente a narrativa e validar o fluxo de ações descritas por ela. O método utilizado para a modelagem nesse trabalho é o proposto por \citeonline{barreto2015modelagem}. Nessa abordagem, os autores representam o jogo por meio de uma rede de Petri que modela uma processo de negócio. Há então um único lugar de início e um de fim; os estados do jogo são representados pelos lugares da redes; as ações/atividades são representadas pelas transições; já o \textit{token} representa o jogador.

A narrativa gerada foi submetida a um processo de decomposição estrutural, no qual as situações descritas foram convertidas em lugares e os eventos ou ações responsáveis por alterações de estado foram representados como transições. Essa correspondência permitiu traduzir a progressão narrativa para um modelo em rede de Petri, evidenciando as relações de causalidade e a coerência entre as etapas da história. O resultado é uma representação formal capaz de demonstrar, o encadeamento lógico presente na narrativa. \citeonline{barreto2015modelagem} ainda cita que o modelo pode ser criado utilizando o software de edição e simulação \textit{CPN Tools} e que pode ser analisado a partir de um método que deriva da verificação da propriedade \textit{soundness}.

As figuras \ref{rede1}, \ref{rede2} e \ref{rede3} apresentam a rede de Petri elaborada a partir da narrativa correspondente gerada pela Inteligência Artificial (Anexo \ref{anexo:narrativa}) a partir do \textit{prompt} (Anexo \ref{anexo:prompt}), representando graficamente a progressão dos eventos e a lógica causal que estrutura o enredo. Esse tipo de modelagem permite compreender a narrativa não apenas em termos de sequência linear, mas como um sistema dinâmico de estados e transições, no qual as ações e reações dos personagens determinam o fluxo da história. Os lugares foram nomeados de P1 a P19. Cada um deles corresponde a uma situação ou estado narrativo, enquanto que as transições interligam essas etapas, indicando a passagem de um estado a outro mediante a ocorrência de um evento ou decisão.

\begin{figure}[H]
    \centering
    \caption{Parte 1 da rede de Petri correspondente a narrativa gerada}
    \includegraphics[width=0.95\linewidth]{imagens/pfc v1 parte 1.png}
    \caption*{}
    \label{rede1}
\end{figure}

A primeira parte da rede, que é ilustrada pela figura \ref{rede1}, compreende os lugares de P1 a P6 e descreve os eventos iniciais que situam o jogador dentro da narrativa. O jogo se inicia com a percepção de que o padre desapareceu (P2), evento que desperta a curiosidade e estabelece o ponto de partida da trama. A partir daí, o jogador inicia a busca por respostas. Neste ponto ele tem a opção de buscar proteção ou investigar, descendo ao porão e examinando o ambiente, o que amplia a sensação de tensão e reforça a atmosfera de mistério. 

A ação de procurar o padre leva à transição para o primeiro momento de ameaça, culminando no aviso de perigo, que simboliza a quebra da normalidade e o ingresso definitivo na zona de conflito. Essa primeira sequência, portanto, estabelece a base dramática e os primeiros elos de causa e efeito que impulsionarão o restante da narrativa, além de apresentar o primeiro contato do jogador com o elemento sobrenatural.

\begin{figure}[H]
    \centering
    \caption{Parte 2 da rede de Petri correspondente a narrativa gerada}
    \includegraphics[width=0.95\linewidth]{imagens/pfc v1 parte 2.png}
    \caption*{}
    \label{rede2}
\end{figure}

A segunda parte da rede pode ser vista na figura \ref{rede2}, que abrange os lugares de P7 a P13, correspondendo ao desenvolvimento e à escalada do conflito. Aqui, o foco se desloca da investigação para a sobrevivência e o impacto psicológico das experiências vividas. O ataque (P8) e a consequente perda de consciência (P9) funcionam como clímax dessa primeira fase, levando a um salto temporal que se manifesta quando o personagem desperta após três dias (P10).

A partir desse ponto, o jogador experimenta uma sensação de desorientação (P11), voltando para casa (P12) e começando a perceber alterações cognitivas e comportamentais (P13). Esses estados indicam a deterioração da estabilidade emocional e marcam a transição do horror físico para o horror psicológico. O fluxo da rede de Petri nesta parte demonstra a sobreposição entre as ações externas e o estado interno do protagonista, reforçando o caráter simbólico da experiência.

\begin{figure}[H]
    \centering
    \caption{Parte 3 da rede de Petri correspondente a narrativa gerada}
    \includegraphics[width=0.95\linewidth]{imagens/pfc v1 parte 3.png}
    \caption*{}
    \label{rede3}
\end{figure}

Por fim, a terceira e última parte da rede de Petri (figura \ref{rede3}) compreende os lugares de P14 a P19 e representa o desfecho da narrativa. Aqui, o protagonista retorna ao convívio familiar, encontrando a mãe (P14). Neste ponto ele tem duas alternativas: tentar estabelecer algum senso de normalidade conversando com a mãe (P15) e controlando seus instintos, ou ele pode fugir para o quarto (P16) e não fazer nada. Entretanto, a progressão dos eventos revela que o personagem está cada vez mais dominado pelos efeitos de suas vivências anteriores.

As ações subsequentes evidenciam o isolamento e o colapso psicológico. A última ação, olhar no espelho, atua como símbolo de reconhecimento e confronto interno, encerrando o ciclo narrativo de forma ambígua e perturbadora. Assim, essa parte final da rede representa o fechamento da jornada, sem necessariamente restaurar o equilíbrio. O modelo reforça a ideia de que, em narrativas de horror psicológico, o conflito central não é completamente resolvido, mas transformado em um estado permanente de incerteza.

A rede de Petri apresentada foi elaborada com o objetivo de representar, de forma visual, a estrutura narrativa correspondente aos dois primeiros momentos da história: a exposição e o incidente incitante. Esses elementos iniciais são fundamentais para o estabelecimento da coerência narrativa e servem como base para a validação dos critérios apresentados na figura \ref{criterios}.

Ao modelar a narrativa em uma rede de Petri, é possível identificar com clareza como cada ação ou evento contribui para a construção da ambientação, para a introdução dos personagens e para o surgimento do evento que desencadeia a trama principal. Nesse sentido, a rede não apenas descreve o fluxo narrativo, mas também evidencia a lógica causal entre as ações, permitindo verificar se a sequência dos acontecimentos respeita a progressão esperada para este estágio da narrativa.

Nos primeiros lugares (P1–P6), a rede contempla a fase da exposição, em que são apresentados o cenário, o protagonista e as motivações iniciais que situam o jogador dentro do universo narrativo. Cada transição representa uma ação introdutória que contribui para a construção da ambientação e para o reconhecimento da situação inicial. Esses elementos correspondem diretamente aos critérios de exposição, ambientação e apresentação do protagonista indicados na tabela de avaliação.

A segunda parte (P7–P13) representa o momento do incidente incitante, isto é, o ponto de ruptura que transforma o estado inicial e impulsiona a narrativa. As transições desta etapa marcam o início do conflito central e indicam uma mudança significativa no estado do personagem. Essa mudança valida a presença de um conflito inicial, bem como a coerência entre as ações e a motivação do protagonista.

Por fim, a última sequência (P14–P19) mantém o foco nas consequências diretas do incidente incitante, reforçando o tom psicológico e a atmosfera de horror propostos no gênero. As ações de fuga, isolamento e confronto interno demonstram que, mesmo sem alcançar o desenvolvimento ou a resolução do enredo, a narrativa estabelece uma base de tensão e imersão, essenciais para a continuidade da história.

Assim, a rede de Petri funciona como uma ferramenta de verificação da estrutura narrativa, permitindo observar de forma objetiva como os elementos da exposição e do incidente incitante se articulam. Ao relacionar esse modelo com os critérios apresentados na tabela \ref{criterios}, é possível confirmar que a narrativa gerada pela IA segue uma progressão lógica e consistente, respeitando as etapas iniciais do paradigma cinematográfico adotado como referência teórica neste trabalho.

\begin{figure}[H]
    \centering
    \caption{Limitabilidade}
    \includegraphics[width=0.95\linewidth]{imagens/boundness.png}
    \caption*{}
    \label{limitabilidade}
\end{figure}

De acordo com \citeonline{barreto2015modelagem}, após a criação do modelo em rede de Petri do jogo, é possível analisá-lo por meio de propriedades bem conhecidas das redes de Petri. Os autores utilizam a análise automática da funcionalidade análise do \textit{state space}, gerada automaticamente pelo \textit{CPN Tools}. Essa funcionalidade verifica automaticamente se a rede submetida possui propriedades estruturais como a limitabilidade e vivacidade. \citeonline{barreto2015modelagem} aponta que essas duas propriedades são suficientes para garantir se o modelo analisado está correto, em termos de execução das atividades do jogo. A figura \ref{limitabilidade} ilustra uma parte do relatório gerado pelo \textit{state space} após a análise do modelo da narrativa. É possível notar que todos os lugares da rede nunca possuem mais de um \textit{token} ao mesmo tempo. Isso indica que o modelo é seguro, pois garante que não ocorre acúmulo indevido ou geração descontrolada de \textit{tokens}, preservando a coerência do fluxo e evitando erros estruturais que poderiam comprometer a lógica representada. Um modelo limitado demonstra que cada etapa do processo é executada de forma controlada e que o sistema não cria estados inconsistentes.

\begin{figure}[H]
    \centering
    \caption{Vivacidade}
    \includegraphics[width=0.95\linewidth]{imagens/liveness.png}
    \caption*{}
    \label{vivacidade}
\end{figure}

Já a figura \ref{vivacidade} ilustra a segunda parte do relatório que aponta a respeito da propriedade de vivacidade. O relatório aponta que não existem marcações mortas, tampouco transições mortas. Em outras palavras, o sistema nunca chega a uma situação em que nenhuma transição pode ser executada, o que significaria um \textit{deadlock}. Além disso, não há transições que nunca se tornam habilitadas ao longo da evolução do sistema. Todas as transições são vivas, o que significa que elas participam de alguma trajetória possível dentro do espaço de estados. Essa propriedade confirma que não existem caminhos inatingíveis e que todas as partes da rede desempenham um papel funcional dentro da dinâmica representada.

Como \citeonline{barreto2015modelagem} destaca, a combinação entre limitabilidade e vivacidade é fundamental para garantir que o modelo represente um fluxo narrativo correto e consistente. No presente trabalho, essas propriedades indicam que, do ponto de vista lógico, a narrativa modelada se desenvolve sem interrupções inesperadas e sem comportamentos anômalos que poderiam comprometer a interpretação da história como um processo dinâmico. 

\chapter{Conclusões e Trabalhos Futuros}\label{cap:conclusao}
\section{Introdução}
Este capítulo tem como objetivo apresentar os principais pontos discutidos no trabalho, relacionar os possíveis trabalhos futuros advindos desta pesquisa e avaliar a principal contribuição deste trabalho para a área científica.


\section{Conclusões}

\noindent Este trabalho demonstrou a viabilidade da utilização de IA generativa para a criação de narrativas de jogos coerentes e imersivas, integrando conceitos cinematográficos com as particularidades do \textit{design} de jogos. Através da elaboração de um \textit{prompt} estruturado baseado nos princípios de narrativa de três atos, foi possível orientar modelos de linguagem na geração de tramas consistentes que atendem aos critérios fundamentais de uma boa narrativa.

A aplicação da técnica de \textit{role prompting}, posicionando a IA no papel de um \textit{game designer}, mostrou-se eficaz na produção de conteúdos que não apenas seguem uma estrutura narrativa sólida, mas também incorporam elementos essenciais, tais como desenvolvimento de personagens, conflito central e ambientação. A comparação entre diferentes LLMs revelou que o \textit{Deepseek V3} apresentou o melhor desempenho na geração de narrativas com riqueza descritiva e coerência estrutural.

Este trabalho oferece contribuições significativas para a área de desenvolvimento de jogos digitais e Inteligência Artificial, estabelecendo pontes metodológicas entre essas áreas e propondo abordagens inovadoras para desafios criativos e técnicos. A abordagem proposta por esta pesquisa oferece uma metodologia estruturada para o desenvolvimento narrativo, a possibilidade de validação formal das estruturas narrativas e propondo o uso da engenharia de \textit{prompt} especializada para narrativas de jogos, funcionando como uma ponte interdisciplinar.

\section{Trabalhos futuros}

\noindent Com base nos resultados e limitações identificadas neste trabalho, sugere-se as seguintes direções para pesquisas futuras:

\begin{itemize}
    \item \textbf{Expansão para narrativas não-lineares:} desenvolver e testar \textbf{prompts} que orientem a geração de tramas ramificadas, onde as decisões do jogador impactam significativamente o desenrolar da história, utilizando redes de Petri mais complexas para modelar e validar múltiplos caminhos narrativos.
    \item \textbf{Integração com mecânicas de jogo:} explorar a geração conjunta de narrativa e mecânicas de \textbf{gameplay}, investigando como a IA pode criar histórias que se integrem organicamente com os sistemas interativos do jogo.
    \item \textbf{Emprego da metodologia em um ambiente de desenvolvimento real:} conduzir um estudo de caso aplicando a metodologia proposta em um estúdio de desenvolvimento de jogos real.

\end{itemize}
\section{Considerações finais}

Este trabalho demonstrou a viabilidade de uma abordagem metodológica que integra IA generativa e modelagem formal para a criação e validação de narrativas de jogos. Os resultados obtidos indicam que é possível, através de um \textit{prompt} estruturado e especializado, orientar LLMs na geração de tramas coerentes e imersivas que atendem aos critérios fundamentais da teoria narrativa.

Do ponto de vista prático, a metodologia desenvolvida oferece aos desenvolvedores de jogos um roteiro sistemático para a criação de narrativas, potencialmente acelerando as fases iniciais de pré-produção e permitindo a exploração de múltiplas direções criativas. A combinação entre a capacidade generativa da IA e a precisão analítica das redes de Petri representa um avanço significativo na busca por ferramentas que ampliem, sem substituir, a criatividade humana no desenvolvimento de jogos.

No entanto, embora os resultados sejam promissores, reconhece-se que a abordagem proposta se beneficia significativamente do conhecimento prévio do usuário tanto em teoria narrativa quanto no funcionamento de sistemas de IA. A qualidade das saídas está diretamente relacionada com a qualidade e especificidade das instruções fornecidas, reforçando que a IA atua como amplificadora, e não substituta, da \textit{expertise} humana.

Em síntese, este trabalho contribui para o avanço do estado da arte no desenvolvimento de jogos ao demonstrar que a integração entre IA generativa e o desenvolvimento de especificação de jogos. Além disso, os critérios estabelecidos para uma boa narrativa ajudam a avaliar qualitativamente as saídas obtidas. A pesquisa proposta neste trabalho abre caminho para novas pesquisas e aplicações práticas que explorem todo o potencial desta abordagem no cenário cada vez mais complexo e competitivo da indústria de \textit{games}.


% ----------------------------------------------------------
% ELEMENTOS PÓS-TEXTUAIS
% ----------------------------------------------------------
\postextual
% ----------------------------------------------------------

% ----------------------------------------------------------
% Referências bibliográficas
% ----------------------------------------------------------
\bibliography{bib}


% ----------------------------------------------------------
% Apêndices
% ----------------------------------------------------------


% ----------------------------------------------------------
% Anexos
% ----------------------------------------------------------

% ---
% Inicia os anexos
% ---
% ---
% Inicia os anexos
% ---
\begin{anexosenv}

% Imprime uma página indicando o início dos anexos
\partanexos

\chapter{Prompt de Exemplo}\label{anexo:prompt}

\noindent A estrutura narrativa de um filme, normalmente, segue o seguinte padrão:

\begin{itemize}
    \item Exposição: Apresentação do mundo, personagens e situação inicial]
    \item Incidente incitante: Evento que desencadeia a história principal
    \item Desenvolvimento/Confrontação: Escalada do conflito e obstáculos
    \item Clímax: Momento de maior tensão e decisão crucial
    \item Resolução/Desfecho: Conclusão dos conflitos e nova situação
\end{itemize}

A trama envolve uma gama de personagens, sendo eles o protagonista, o antagonista e os personagens secundários. É comum que haja algum tipo de arco ou curva de personagem mostrando uma grande mudança no protagonista ou antagonista. Pode ser um arco de redenção, a jornada do herói, etc.

A narrativa, por sua vez, é movida por um conflito, que pode envolver vários personagens ou ser um conflito interno/psicológico do personagem principal.

É comum que haja algum tipo de tema central ou subtexto abordado, dando mais profundidade a história e aos personagens abordados.

Outro elemento extremamente importante é a ambientação: tempo e espaço, onde e quando a história se passa.

Sabe-se que atualmente as narrativas de jogos têm se tornado cada vez mais cinematográficas. Isso permite que, ao delimitar os elementos essenciais para uma narrativa de jogo, conceitos do cinema possam ser amplamente utilizados.

Tendo tudo isso em mente:

Você é um \textit{game designer} que cria roteiros/narrativas para jogos. Sua missão é criar uma narrativa para um jogo de terror, sobre vampiros, com um protagonista masculino de origem brasileira com ascendência sul coreana, chamado Shin. Lembre: o protagonista deve ter um perfil definido (nome, origem, histórico, motivação). Atenha-se às etapas 1 e 2 do padrão que o cinema utiliza. Lembre-se de fazer um planejamento, apresentando o personagem e sua história de background, como uma espécie de perfil, e depois escreva a história de uma fase inicial e introdutória do jogo, que, em um filme, seria como a cena de abertura + parte do primeiro ato. O tom deve ser <atmosfera>.

Saída esperada (com títulos e seções):

Perfil do Protagonista – apresentação detalhada.

Exposição (Abertura do jogo) – cena inicial + contexto do mundo.

Incidente Incitante – evento que inicia a jornada do jogador.

(A narrativa deve ter abertura para decisões ou ações do jogador, ou seja, deve ser interativa, a menos que seja uma cut scene.)

História base: Shin é um jovem religioso de 14 anos. Ele ama a igreja e quer começar a se preparar para ser padre. Um dia, devido a uma forte tempestade, ele fica preso dentro da igreja que frequentava e no porão de lá ele é atacado por uma criatura desconhecida. Três dias depois ele acorda com uma sede incontrolável e completamente confuso. Ele volta para a casa dos pais, que o estavam procurando pelo tempo que passou sumido. Ele fica cada vez mais debilitado com o passar dos dias, com aparência cadavérica, presas proeminentes, íris vermelhas e uma estranha urgência de (em itálico) rasgar a pele dos humanos.

Escreva, em forma de roteiro, detalhando exatamente em quais partes o jogador deve tomar decisões e quais são todas as ações do personagem, a partir disso, detalhando com riqueza os acontecimentos e os destrinchando em ações e decisões que impactem a narrativa a serem tomadas.

\chapter{Narrativa Gerada pelo \textit{Deepseek}}\label{anexo:narrativa}

\noindent SANGUIS ET FIDES: Roteiro de Jogo - Fase Inicial

Perfil do Protagonista

Nome: Shin Kim
Idade: 14 anos
Origem: Curitiba, Paraná, Brasil. Mãe brasileira, pai sul-coreano.
Aparência: Cabelos negros lisos, olhos castanhos amendoados, estatura mediana para a idade. Após o incidente, sua pele se tornará pálida e translúcida, seus olhos desenvolverão uma íris vermelha e suas presas caninas se tornarão proeminentes.

Histórico e Personalidade:
Shin é um jovem tranquilo, introspectivo e profundamente religioso. Cresceu entre duas culturas: a calorosidade e a fé católica da mãe brasileira e a disciplina e o respeito às tradições do pai coreano. A igreja sempre foi seu refúgio, um lugar de ordem e paz onde ele sentia que pertencia. Sua maior ambição é entrar para o seminário assim que completar 18 anos. Ele é observador, gentil e tem um forte senso de dever, mas carrega uma certa ingenuidade típica de quem passou a vida dentro de um ambiente protegido.

Motivação Inicial: Servir a Deus, encontrar seu lugar no mundo e fazer seus pais orgulhosos.
Motivação Pós-Incidente (Arco Previsto): Sobreviver, entender a maldição que o consome e, acima de tudo, lutar contra a besta que agora habita seu corpo, buscando uma cura ou uma forma de redimir sua alma perante a fé que sempre guiou sua vida. O conflito central será interno: sua devoção religiosa contra a sede vampírica incontrolável.

Narrativa da Fase Inicial e Introdutória

1. Exposição (Abertura do jogo)

CENA 1 - Cut Scene (Não interativa)

[CENA SE INICIA]

AMBIENTAÇÃO: Noite de temporal em Curitiba. A câmera mostra uma vista exterior da Igreja de São Sebastião, uma construção antiga de pedra. A chuva cai torrencialmente, os raios iluminam o céu, e os ventos forte sacodem os galhos das árvores do cemitério anexo.

INTERIOR DA IGREJA - NOITE

A câmera adentra a igreja, vazia. A iluminação é fraca, vinda de velas tremulantes nos altares. O som da chuva batendo nos vitrais coloridos é alto e ameaçador. SHIN, de 14 anos, veste uma calça jeans e um casaco sobre a camisa social. Ele está ajoelhado no primeiro banco, diante do altar-mor, rezando em voz baixa. Seu rosto é sereno, iluminado pelas chamas das velas.

PADRE EMÍLIO (V.O., com voz cansada e gentil): "Shin, meu filho. Já está muito tarde. E essa tempestade não vai dar trégua."

Shin se vira. O PADRE EMÍLIO (60 anos), um homem bondoso que é seu mentor, se aproxima.

SHIN: "Só terminei de arrumar os missais, Padre. E queria pedir forças para a conversa com meus pais... sobre o seminário."

PADRE EMÍLIO: (Sorrindo) "Eles te amam, Shin. Vão entender. Agora vá para casa. Vou fechar a igreja."

Shin se levanta, pega sua mochila e se dirige para a porta principal. Ele a empurra, mas um forte vendaval a empurra de volta, fechando-a com força.

PADRE EMÍLIO: "Pelos céus... Nunca vi algo assim. O portão do cemitério deve ter arrombado. Vou ver a saída lateral. A energia caiu, o gerador vai acionar no porão. Fique aqui."

O padre pega um velho candelabro com várias velas e se dirige para uma porta de madeira pesada, ao lado do altar, que leva ao subsolo. A câmera foca em Shin, sozinho na nave da igreja. O silêncio é quebrado apenas pelo uivar do vento e por um **SOM BAIXO E METÁLICO**, como se algo tivesse se soltado no porão.

SHIN: (Para si mesmo) "Padre Emílio?"

Não há resposta. Um barulho de queda abafada vem do porão. As velas da igreja tremulam violentamente.

[FIM DA CUT SCENE]


MODO JOGO: Controle do jogador é liberado.


AÇÃO DO JOGADOR:
Shin está no centro da nave da igreja. O jogador pode controlá-lo.

Objetivo: Investigar o silêncio do Padre Emílio.
Ambiente Interativo:
O jogador pode andar pela igreja.
As velas projetam sombras dinâmicas e assustadoras.
O som da tempestade é constante.
A porta do porão está entreaberta, com uma luz fraca e trêmula (do gerador) vindo de dentro.

DECISÃO 1: O que Shin deve fazer?

Opção A (Fé/Coragem): "Preciso ver se o Padre está bem." (Dirigir-se diretamente à porta do porão).
Opção B (Cautela/Medo): "Algo não está certo. Melhor procurar algo para me defender." (Procurar um objeto na igreja antes de descer. Ex: um crucifixo grande de prata perto do batistério, ou um dos castiçais de metal do altar).

CONSEQUÊNCIA DA DECISÃO:
Escolher a Opção B adiciona um momento de tensão e permite que Shin pegue um item (ex: Castiçal de Metal), que pode ser usado em um evento de *Quick Time Event* (QTE) mais tarde, dando uma pequena vantagem. A narrativa principal, no entanto, prossegue de forma similar.


2. Incidente Incitante

SEQUÊNCIA INTERATIVA: O Porão

Independente da escolha, Shin desce as escadas de pedra estreitas que levam ao porão.

AMBIENTAÇÃO: O porão é úmido, abafado e cheira a mofo e óleo. O gerador a diesel ronca em um canto, iluminando o ambiente de forma irregular. Estantes com documentos antigos da paróquia e barris vazios se alinham nas paredes. No centro do cômodo, o candelabro do Padre Emílio está caído no chão, as velas apagadas. O padre não está visível.

DIÁLOGO INTERNO (Voz do Shin, sussurrada, acionada ao se aproximar do candelabro):** "Onde ele está? Deus, me dê coragem..."

AÇÃO DO JOGADOR:
Shin deve investigar o porão.

Interagir com o Candelabro Caído: "As velas se apagaram como se um vento forte tivesse passado... mas aqui não há vento."*
Interagir com uma estante tombada: "Parece que alguém... ou algo... empurrou isso com violência."
Ouvir um som: Um arrastar fraco e um gemido abafado vêm de um canto escuro, atrás de uma pilha de barris.

Shin se aproxima cautelosamente. Ao virar o último barril, ele encontra o Padre Emílio, pálido e ensanguentado, escondido.

PADRE EMÍLIO (Ofegante, com voz fraca e aterrorizada): "Fuja, Shin! Não é um ladrão... É uma criatura... da escuridão pura! Ela está aqui!"

Antes que Shin possa reagir, as luzes do gerador **piscam e se apagam**, mergulhando o porão na escuridão quase total. Por um instante, o gerador falha. Quando a luz volta, titubeante, uma figura alta e pálida, de movimentos fluidos e antinaturais, está parada a alguns metros de distância.

A CRIATURA: Sua pele é branca como a porcelana, seus olhos são completamente negros e um sorriso largo e cheio de dentes afiados se estende em seu rosto. Suas roupas são escuras e esvoaçantes.

QTE (Quick Time Event): A Investida
A criatura se move num borrão de velocidade em direção a Shin.

Se o jogador escolheu a Opção B (Castiçal): Um prompt na tela ordena **[BOTÃO X]**. Ao acertar, Shin levanta o castiçal instintivamente. A criatura desvia com um silvo de desdém, mas a investida é interrompida. Ela então ataca com uma força sobre-humana, arremessando Shin contra a parede.
Se o jogador escolheu a Opção A (Sem item): Um prompt desafiador de [BOTÃO X] aparece. Se falhar, ou se não tiver o item, a criatura atinge Shin diretamente, arremessando-o contra a parede.

CUT SCENE (Não interativa)

Shin cai no chão, atordoado. A criatura se aproxima, ignorando completamente o Padre Emílio, que grita em desespero. Ela se inclina sobre Shin, que tenta se esquivar inutilmente.

CRIATURA (Com voz sibilante e sedosa): "Um filho da fé... tão jovem... tão puro. Um vaso perfeito."

A criatura segura Shin com uma força irreprimível. Ele sente uma dor aguda e gelada no pescoço. Sua visão escurece. A última coisa que ele vê é o rosto aterrorizado do Padre Emílio antes de perder a consciência.

[TELA FICA PRETA]

SONS: Chuva, um grito abafado do padre, e depois... silêncio.

[FADE IN]

CENA 3 - O Despertar

AMBIENTAÇÃO: A câmera está desfocada. Shin acorda deitado no banco de madeira da própria igreja. A luz do amanhecer entra suavemente pelos vitrais. A tempestade passou. Ele está sozinho.

AÇÃO DO JOGADOR:
Shin se levanta, sentindo-se extremamente confuso e fraco.

Objetivo: Voltar para casa.
Sensações (Feedback de Áudio e Visual):
A câmera está levemente instável, simulando desorientação.
Sons normais são distorcidos e amplificados: o arrastar de seus pés no chão soa como um trovão, o canto de um pássaro lá fora é agudo e doloroso.
Visão: As cores estão dessaturadas, mas o mundo parece incrivelmente nítido. Shin vê poeira flutuando no ar com detalhes nunca percebidos antes.
Sede: Uma *"Barra de Sede"* aparece discretamente na HUD, já no vermelho. É uma sensação avassaladora de secura, como se sua garganta estivesse fechada.

Shin caminha com dificuldade até sua casa, que fica a poucas quadras da igreja. O caminho é uma provação. Ele vê pessoas na rua, e um novo elemento surge:

IMPULSO (Feedback Visual): Quando Shin olha para a nuca de um transeunte, uma dica visual sutil (como um contorno vermelho pulsante) aparece em volta da pessoa, acompanhada de um sussurro na mente de Shin: *"Rasgar... sugar... aliviar a sede..."*

DECISÃO 2: O primeiro contato.

Shin chega em casa. Sua MÃE está na sala, com o rosto marcado pela preocupação.

MÃE: "SHIN! Meu Deus, onde você esteve? Sumido por três dias! Seu pai está pela cidade toda te procurando!"

Shin tenta responder, mas a visão da veia pulsando no pescoço de sua mãe domina sua atenção. A *Barra de Sede* pisca.

Opção A (Controle): "Mãe, eu... não me sinto bem." (Shin desvia o olhar, focando no chão. Ele luta contra o impulso).
Opção B (Fuga): (Sem diálogo) Shin ignora a mãe e corre em direção ao seu quarto, trancando a porta.

CONSEQUÊNCIA:
Ambas as opções levam ao mesmo resultado narrativo: Shin se tranca no quarto, aterrorizado com o que está sentindo. A **Opção A** mostra sua luta interna de forma mais dramática.

CUT SCENE FINAL DA INTRODUÇÃO

Shin se olha no espelho do quarto. Sua aparência está cadavérica, pálida. Suas íris agora são de um vermelho sangue. Ele levanta o lábio superior e vê duas presas afiadas e proeminentes onde antes estavam seus caninos. Ele abre a boca para gritar, mas só sai um sussurro rouco de terror. A câmera se aproxima do seu rosto refletido, seus olhos vermelhos cheios de pavor e confusão.

TELA PRETA.

TÍTULO DO JOGO: SANGUIS ET FIDES

A fase inicial termina aqui, com o Incidente Incitante (o ataque no porão) concluído e a jornada de horror pessoal de Shin oficialmente iniciada. O jogador entende o conflito: controlar um personagem que precisa lutar contra sua própria natureza bestial para salvar sua alma.




\end{anexosenv}

%---------------------------------------------------------------------
% INDICE REMISSIVO
%---------------------------------------------------------------------
%\phantompart
\printindex
%---------------------------------------------------------------------

\end{document}
