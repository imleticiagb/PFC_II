% ---
% RESUMOS
% ---

% resumo em português
\setlength{\absparsep}{18pt} % ajusta o espaçamento dos parágrafos do resumo
\begin{resumo}

Este trabalho investiga o uso de Inteligência Artificial generativa como ferramenta para criação de narrativas de jogos digitais, examinando se modelos de linguagem de grande escala (LLMs) são capazes de produzir histórias coerentes, cinematográficas e adequadas às necessidades do \textit{game design}. O problema central reside na dificuldade de gerar narrativas que mantenham coesão interna, progressão dramática e lógica causals. O objetivo principal é propor e avaliar um método que permita tanto a geração quanto a validação formal dessas narrativas, estabelecendo critérios estruturais baseados nos estudos de cinema e nos princípios narrativos aplicados ao desenvolvimento de jogos. A pesquisa justifica-se pela crescente presença da IA no processo criativo, pela demanda por métodos acessíveis para desenvolvedores independentes e pela escassez de estudos que integrem técnicas narrativas, modelos generativos e ferramentas formais de validação. Metodologicamente, este trabalho parte da definição de elementos essenciais de uma boa narrativa e da construção de um \textit{prompt} detalhado baseado em personificação (\textit{role prompting}). Esse \textit{prompt} foi aplicado em três modelos de IA amplamente utilizados, sendo eles: \textit{GPT-5}, \textit{DeepSeek V3} e \textit{Gemini 2.5 Flash}. O objetivo era gerar narrativas comparáveis. As produções foram analisadas a partir de critérios formais estabelecidos na literatura e, posteriormente, traduzidas para um modelo de rede de Petri, a fim de verificar propriedades como coerência estrutural, fluxo lógico e progressão causal. Os resultados demonstram que a IA é capaz de produzir narrativas consistentes quando orientada por um \textit{prompt} bem estruturado, sendo o modelo \textit{DeepSeek V3} o que apresentou melhor aderência aos critérios definidos. A modelagem em rede de Petri confirmou a possibilidade de validação formal de narrativas geradas automaticamente, evidenciando sua estrutura lógica. Como contribuição, esta pesquisa propõe um roteiro metodológico replicável e acessível, que integra engenharia de \textit{prompt} e validação formal, oferecendo uma abordagem inédita para o uso de IA generativa na concepção narrativa de jogos.

 \textbf{Palavras-chaves}: \textit{Inteligência Artificial Generativa; Narrativas para Jogos; Redes de Petri; Modelagem; Game Design.} 


\end{resumo}

% resumo em inglês
\begin{resumo}[Abstract]
 \begin{otherlanguage*}{english}
 This work investigates the use of generative Artificial Intelligence as a tool for the creation of digital game narratives, examining whether large language models (LLMs) are capable of producing coherent, cinematic stories that fit the needs of game design. The central problem lies in the difficulty of generating narratives that maintain internal cohesion, dramatic progression, and causal logic. The main objective is to propose and evaluate a method that allows both the generation and formal validation of these narratives, establishing structural criteria based on film studies and narrative principles applied to game development. The research is justified by the growing presence of AI in the creative process, the demand for accessible methods for independent developers, and the scarcity of studies that integrate narrative techniques, generative models, and formal validation tools. Methodologically, this work begins with the definition of essential elements of a good narrative and the construction of a detailed prompt based on personification (role prompting). This prompt was applied to three widely used AI models: GPT-5, DeepSeek V3, and Gemini 2.5 Flash. The objective was to generate comparable narratives. The outputs were analyzed using formal criteria established in the literature and later translated into a Petri net model in order to verify properties such as structural coherence, logical flow, and causal progression. The results demonstrate that AI is capable of producing consistent narratives when guided by a well-structured prompt, with the DeepSeek V3 model showing the best adherence to the defined criteria. The Petri net modeling confirmed the possibility of formal validation of automatically generated narratives, highlighting their logical structure. As a contribution, this research proposes a replicable and accessible methodological framework that integrates prompt engineering and formal validation, offering a novel approach to the use of generative AI in narrative design for games.
   
   \textbf{Key-words}: 
   \textit{Generative Artificial Intelligence; Game Narratives; Petri Nets; Modeling; Game Design.}
 \end{otherlanguage*}
\end{resumo}

